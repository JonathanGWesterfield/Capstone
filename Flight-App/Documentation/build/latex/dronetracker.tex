%% Generated by Sphinx.
\def\sphinxdocclass{report}
\documentclass[letterpaper,10pt,english]{sphinxmanual}
\ifdefined\pdfpxdimen
   \let\sphinxpxdimen\pdfpxdimen\else\newdimen\sphinxpxdimen
\fi \sphinxpxdimen=.75bp\relax

\PassOptionsToPackage{warn}{textcomp}
\usepackage[utf8]{inputenc}
\ifdefined\DeclareUnicodeCharacter
% support both utf8 and utf8x syntaxes
  \ifdefined\DeclareUnicodeCharacterAsOptional
    \def\sphinxDUC#1{\DeclareUnicodeCharacter{"#1}}
  \else
    \let\sphinxDUC\DeclareUnicodeCharacter
  \fi
  \sphinxDUC{00A0}{\nobreakspace}
  \sphinxDUC{2500}{\sphinxunichar{2500}}
  \sphinxDUC{2502}{\sphinxunichar{2502}}
  \sphinxDUC{2514}{\sphinxunichar{2514}}
  \sphinxDUC{251C}{\sphinxunichar{251C}}
  \sphinxDUC{2572}{\textbackslash}
\fi
\usepackage{cmap}
\usepackage[T1]{fontenc}
\usepackage{amsmath,amssymb,amstext}
\usepackage{babel}



\usepackage{times}
\expandafter\ifx\csname T@LGR\endcsname\relax
\else
% LGR was declared as font encoding
  \substitutefont{LGR}{\rmdefault}{cmr}
  \substitutefont{LGR}{\sfdefault}{cmss}
  \substitutefont{LGR}{\ttdefault}{cmtt}
\fi
\expandafter\ifx\csname T@X2\endcsname\relax
  \expandafter\ifx\csname T@T2A\endcsname\relax
  \else
  % T2A was declared as font encoding
    \substitutefont{T2A}{\rmdefault}{cmr}
    \substitutefont{T2A}{\sfdefault}{cmss}
    \substitutefont{T2A}{\ttdefault}{cmtt}
  \fi
\else
% X2 was declared as font encoding
  \substitutefont{X2}{\rmdefault}{cmr}
  \substitutefont{X2}{\sfdefault}{cmss}
  \substitutefont{X2}{\ttdefault}{cmtt}
\fi


\usepackage[Bjarne]{fncychap}
\usepackage{sphinx}

\fvset{fontsize=\small}
\usepackage{geometry}

% Include hyperref last.
\usepackage{hyperref}
% Fix anchor placement for figures with captions.
\usepackage{hypcap}% it must be loaded after hyperref.
% Set up styles of URL: it should be placed after hyperref.
\urlstyle{same}
\addto\captionsenglish{\renewcommand{\contentsname}{Contents:}}

\usepackage{sphinxmessages}
\setcounter{tocdepth}{2}



\title{Drone Tracker}
\date{Nov 18, 2019}
\release{1.0.3}
\author{Hayley Eckert, Jonathan Westerfield, Donald Elrod, Ismael Rodriguez}
\newcommand{\sphinxlogo}{\vbox{}}
\renewcommand{\releasename}{Release}
\makeindex
\begin{document}

\pagestyle{empty}
\sphinxmaketitle
\pagestyle{plain}
\sphinxtableofcontents
\pagestyle{normal}
\phantomsection\label{\detokenize{index::doc}}



\chapter{Program Controller}
\label{\detokenize{index:module-src.Controllers.Program_Controller}}\label{\detokenize{index:program-controller}}\index{src.Controllers.Program\_Controller (module)@\spxentry{src.Controllers.Program\_Controller}\spxextra{module}}\index{Controller (class in src.Controllers.Program\_Controller)@\spxentry{Controller}\spxextra{class in src.Controllers.Program\_Controller}}

\begin{fulllineitems}
\phantomsection\label{\detokenize{index:src.Controllers.Program_Controller.Controller}}\pysiglinewithargsret{\sphinxbfcode{\sphinxupquote{class }}\sphinxcode{\sphinxupquote{src.Controllers.Program\_Controller.}}\sphinxbfcode{\sphinxupquote{Controller}}}{\emph{phoneControl: Controllers.PhoneController.PhoneControl}}{}
Controller class for the application. Changes between application views based on user input.
Run this file in order to begin the application.
\index{cleanup() (src.Controllers.Program\_Controller.Controller method)@\spxentry{cleanup()}\spxextra{src.Controllers.Program\_Controller.Controller method}}

\begin{fulllineitems}
\phantomsection\label{\detokenize{index:src.Controllers.Program_Controller.Controller.cleanup}}\pysiglinewithargsret{\sphinxbfcode{\sphinxupquote{cleanup}}}{}{{ $\rightarrow$ None}}
Goes through our file structure and deletes all files (except anything in the Flight folder).
This is for when we are done with the program and want to delete the raw footage from the phone,
the output points from the opencv processing and any other files that were used during the
program execution. Should also be called before any time the user wants to fly.
\begin{quote}\begin{description}
\item[{Returns}] \leavevmode
None

\end{description}\end{quote}

\end{fulllineitems}

\index{get\_all\_files() (src.Controllers.Program\_Controller.Controller method)@\spxentry{get\_all\_files()}\spxextra{src.Controllers.Program\_Controller.Controller method}}

\begin{fulllineitems}
\phantomsection\label{\detokenize{index:src.Controllers.Program_Controller.Controller.get_all_files}}\pysiglinewithargsret{\sphinxbfcode{\sphinxupquote{get\_all\_files}}}{\emph{folderPath: str}}{{ $\rightarrow$ list}}
Will get all of the file names in a folder. This will be used after the video footage is transferred from
the phone to the laptop. We need to grab the 2 videos filenames in the folder in order to send them to
the opencv analysis to have the coordinates extracted. This will also be used when we need to splice
together all of the coordinates output by the OpenCVController json files.
\begin{quote}\begin{description}
\item[{Parameters}] \leavevmode
\sphinxstyleliteralstrong{\sphinxupquote{folderPath}} \textendash{} The directory that we need to get all of the files from.

\item[{Returns}] \leavevmode
A list of all of the file names in that directory

\end{description}\end{quote}

\end{fulllineitems}

\index{get\_flight\_info() (src.Controllers.Program\_Controller.Controller method)@\spxentry{get\_flight\_info()}\spxextra{src.Controllers.Program\_Controller.Controller method}}

\begin{fulllineitems}
\phantomsection\label{\detokenize{index:src.Controllers.Program_Controller.Controller.get_flight_info}}\pysiglinewithargsret{\sphinxbfcode{\sphinxupquote{get\_flight\_info}}}{\emph{pilotName: str}, \emph{instructorName: str}, \emph{flightInstructions: str}}{{ $\rightarrow$ None}}
Saves the pilot name, instructor name, and flight instructions once confirmed by the user.
\begin{quote}\begin{description}
\item[{Parameters}] \leavevmode\begin{itemize}
\item {} 
\sphinxstyleliteralstrong{\sphinxupquote{pilotName}} \textendash{} String containing the pilot name

\item {} 
\sphinxstyleliteralstrong{\sphinxupquote{instructorName}} \textendash{} String containing the instructor name

\item {} 
\sphinxstyleliteralstrong{\sphinxupquote{flightInstructions}} \textendash{} String containing the flight instructions

\end{itemize}

\item[{Returns}] \leavevmode
None

\end{description}\end{quote}

\end{fulllineitems}

\index{get\_in\_order() (src.Controllers.Program\_Controller.Controller method)@\spxentry{get\_in\_order()}\spxextra{src.Controllers.Program\_Controller.Controller method}}

\begin{fulllineitems}
\phantomsection\label{\detokenize{index:src.Controllers.Program_Controller.Controller.get_in_order}}\pysiglinewithargsret{\sphinxbfcode{\sphinxupquote{get\_in\_order}}}{\emph{files: list}}{{ $\rightarrow$ list}}
Will return the file names in order. This means that phone-1 will be first in the list and
phone-2 will be second. Need this to ensure that a specific phone’s coordinates are being
used for the Z, X axis and the other is used for Z, Y
\begin{quote}\begin{description}
\item[{Parameters}] \leavevmode
\sphinxstyleliteralstrong{\sphinxupquote{files}} \textendash{} The list of files we get from the get\_all\_files() function

\item[{Returns}] \leavevmode
The list of file names starting with phone-1 first and phone-2 second

\end{description}\end{quote}

\end{fulllineitems}

\index{import\_flight() (src.Controllers.Program\_Controller.Controller method)@\spxentry{import\_flight()}\spxextra{src.Controllers.Program\_Controller.Controller method}}

\begin{fulllineitems}
\phantomsection\label{\detokenize{index:src.Controllers.Program_Controller.Controller.import_flight}}\pysiglinewithargsret{\sphinxbfcode{\sphinxupquote{import\_flight}}}{\emph{flightPath: str}}{{ $\rightarrow$ None}}
Reads in a .flight file and displays the report view for it.
\begin{quote}\begin{description}
\item[{Parameters}] \leavevmode
\sphinxstyleliteralstrong{\sphinxupquote{flightPath}} \textendash{} String with file path of chosen file to import.

\item[{Returns}] \leavevmode
None

\end{description}\end{quote}

\end{fulllineitems}

\index{setupFileStructure() (src.Controllers.Program\_Controller.Controller method)@\spxentry{setupFileStructure()}\spxextra{src.Controllers.Program\_Controller.Controller method}}

\begin{fulllineitems}
\phantomsection\label{\detokenize{index:src.Controllers.Program_Controller.Controller.setupFileStructure}}\pysiglinewithargsret{\sphinxbfcode{\sphinxupquote{setupFileStructure}}}{}{{ $\rightarrow$ None}}
Sets up the folders that need to exist before we can transfer footage and analyze. The file structure
should be: drone-tracker \textgreater{} FTP, opencv-output, Flights. This makes it easier to keep track of
the files that we are working with during the application lifecycle.
\begin{quote}\begin{description}
\item[{Returns}] \leavevmode
None

\end{description}\end{quote}

\end{fulllineitems}

\index{show\_home() (src.Controllers.Program\_Controller.Controller method)@\spxentry{show\_home()}\spxextra{src.Controllers.Program\_Controller.Controller method}}

\begin{fulllineitems}
\phantomsection\label{\detokenize{index:src.Controllers.Program_Controller.Controller.show_home}}\pysiglinewithargsret{\sphinxbfcode{\sphinxupquote{show\_home}}}{}{{ $\rightarrow$ None}}
Loads the home startup screen for the user.
\begin{quote}\begin{description}
\item[{Returns}] \leavevmode
None

\end{description}\end{quote}

\end{fulllineitems}

\index{show\_loading\_window() (src.Controllers.Program\_Controller.Controller method)@\spxentry{show\_loading\_window()}\spxextra{src.Controllers.Program\_Controller.Controller method}}

\begin{fulllineitems}
\phantomsection\label{\detokenize{index:src.Controllers.Program_Controller.Controller.show_loading_window}}\pysiglinewithargsret{\sphinxbfcode{\sphinxupquote{show\_loading\_window}}}{}{{ $\rightarrow$ None}}
Loads the loading screen for the user.
\begin{quote}\begin{description}
\item[{Returns}] \leavevmode
None

\end{description}\end{quote}

\end{fulllineitems}

\index{show\_report\_window() (src.Controllers.Program\_Controller.Controller method)@\spxentry{show\_report\_window()}\spxextra{src.Controllers.Program\_Controller.Controller method}}

\begin{fulllineitems}
\phantomsection\label{\detokenize{index:src.Controllers.Program_Controller.Controller.show_report_window}}\pysiglinewithargsret{\sphinxbfcode{\sphinxupquote{show\_report\_window}}}{\emph{previousFlight: str}, \emph{usingPreviousFlight: bool}, \emph{flightData: dict}}{{ $\rightarrow$ None}}
Loads the report screen for the user.
\begin{quote}\begin{description}
\item[{Parameters}] \leavevmode\begin{itemize}
\item {} 
\sphinxstyleliteralstrong{\sphinxupquote{previousFlight}} \textendash{} String containing path to flight data. Should be .flight file if usingPreviousFlight is true, or empty if usingPreviousFlight is false.

\item {} 
\sphinxstyleliteralstrong{\sphinxupquote{usingPreviousFlight}} \textendash{} Boolean representing if the report view is for an existing .flight file or a new analysis.

\item {} 
\sphinxstyleliteralstrong{\sphinxupquote{flightData}} \textendash{} Dictionary containing the flight data. Should be populated with only coordinates if usingPreviousFlight is false, and empty if usingPreviousFlight is true.

\end{itemize}

\item[{Returns}] \leavevmode
None

\end{description}\end{quote}

\end{fulllineitems}

\index{show\_tracking\_window() (src.Controllers.Program\_Controller.Controller method)@\spxentry{show\_tracking\_window()}\spxextra{src.Controllers.Program\_Controller.Controller method}}

\begin{fulllineitems}
\phantomsection\label{\detokenize{index:src.Controllers.Program_Controller.Controller.show_tracking_window}}\pysiglinewithargsret{\sphinxbfcode{\sphinxupquote{show\_tracking\_window}}}{}{{ $\rightarrow$ None}}
Loads the tracking screen for the user.
\begin{quote}\begin{description}
\item[{Returns}] \leavevmode
None

\end{description}\end{quote}

\end{fulllineitems}

\index{show\_verify\_screen() (src.Controllers.Program\_Controller.Controller method)@\spxentry{show\_verify\_screen()}\spxextra{src.Controllers.Program\_Controller.Controller method}}

\begin{fulllineitems}
\phantomsection\label{\detokenize{index:src.Controllers.Program_Controller.Controller.show_verify_screen}}\pysiglinewithargsret{\sphinxbfcode{\sphinxupquote{show\_verify\_screen}}}{}{{ $\rightarrow$ None}}
Loads the verify setup screen for the user.
\begin{quote}\begin{description}
\item[{Returns}] \leavevmode
None

\end{description}\end{quote}

\end{fulllineitems}

\index{start\_analysis() (src.Controllers.Program\_Controller.Controller method)@\spxentry{start\_analysis()}\spxextra{src.Controllers.Program\_Controller.Controller method}}

\begin{fulllineitems}
\phantomsection\label{\detokenize{index:src.Controllers.Program_Controller.Controller.start_analysis}}\pysiglinewithargsret{\sphinxbfcode{\sphinxupquote{start\_analysis}}}{}{{ $\rightarrow$ None}}
Spawns the sub processes that will analyze the footage of the drone footage. We will need
to have the files before hand so we can pass them into each OpenCVController process.
\begin{quote}\begin{description}
\item[{Returns}] \leavevmode
None

\end{description}\end{quote}

\end{fulllineitems}

\index{transfer\_complete() (src.Controllers.Program\_Controller.Controller method)@\spxentry{transfer\_complete()}\spxextra{src.Controllers.Program\_Controller.Controller method}}

\begin{fulllineitems}
\phantomsection\label{\detokenize{index:src.Controllers.Program_Controller.Controller.transfer_complete}}\pysiglinewithargsret{\sphinxbfcode{\sphinxupquote{transfer\_complete}}}{\emph{flightData: dict}}{{ $\rightarrow$ None}}
Calls the report view using the flight data dictionary.
\begin{quote}\begin{description}
\item[{Parameters}] \leavevmode
\sphinxstyleliteralstrong{\sphinxupquote{flightData}} \textendash{} Dictionary of flight data

\item[{Returns}] \leavevmode
none.

\end{description}\end{quote}

\end{fulllineitems}

\index{transfer\_footage() (src.Controllers.Program\_Controller.Controller method)@\spxentry{transfer\_footage()}\spxextra{src.Controllers.Program\_Controller.Controller method}}

\begin{fulllineitems}
\phantomsection\label{\detokenize{index:src.Controllers.Program_Controller.Controller.transfer_footage}}\pysiglinewithargsret{\sphinxbfcode{\sphinxupquote{transfer\_footage}}}{\emph{phoneControl: Controllers.PhoneController.PhoneControl}}{{ $\rightarrow$ None}}
Transfers footage and calls DroneController to analyze the footage.
\begin{quote}\begin{description}
\item[{Parameters}] \leavevmode
\sphinxstyleliteralstrong{\sphinxupquote{phoneControl}} \textendash{} Phone Controller object for the active phone connection.

\item[{Returns}] \leavevmode
none

\end{description}\end{quote}

\end{fulllineitems}

\index{updateFlightStatus() (src.Controllers.Program\_Controller.Controller method)@\spxentry{updateFlightStatus()}\spxextra{src.Controllers.Program\_Controller.Controller method}}

\begin{fulllineitems}
\phantomsection\label{\detokenize{index:src.Controllers.Program_Controller.Controller.updateFlightStatus}}\pysiglinewithargsret{\sphinxbfcode{\sphinxupquote{updateFlightStatus}}}{}{{ $\rightarrow$ None}}
Sets the status of the system verification test.
\begin{quote}\begin{description}
\item[{Returns}] \leavevmode
none

\end{description}\end{quote}

\end{fulllineitems}

\index{wait\_for\_analysis() (src.Controllers.Program\_Controller.Controller method)@\spxentry{wait\_for\_analysis()}\spxextra{src.Controllers.Program\_Controller.Controller method}}

\begin{fulllineitems}
\phantomsection\label{\detokenize{index:src.Controllers.Program_Controller.Controller.wait_for_analysis}}\pysiglinewithargsret{\sphinxbfcode{\sphinxupquote{wait\_for\_analysis}}}{}{{ $\rightarrow$ None}}
Waits for the analysis of the footage to complete. Essentially is just a loop that checks to
see if the file locks ({\color{red}\bfseries{}*}.lock) for our files are still in that directory. If they are, we wait,
otherwise, we will exit the loop. From there, we need to get the output from those processes and splice
the points together into a single 3D coordinate list.
\begin{quote}\begin{description}
\item[{Returns}] \leavevmode
None

\end{description}\end{quote}

\end{fulllineitems}


\end{fulllineitems}

\index{close\_conn() (in module src.Controllers.Program\_Controller)@\spxentry{close\_conn()}\spxextra{in module src.Controllers.Program\_Controller}}

\begin{fulllineitems}
\phantomsection\label{\detokenize{index:src.Controllers.Program_Controller.close_conn}}\pysiglinewithargsret{\sphinxcode{\sphinxupquote{src.Controllers.Program\_Controller.}}\sphinxbfcode{\sphinxupquote{close\_conn}}}{\emph{phoneControl: Controllers.PhoneController.PhoneControl}}{{ $\rightarrow$ None}}
Closes the connection from the laptop to the phones.
\begin{quote}\begin{description}
\item[{Parameters}] \leavevmode
\sphinxstyleliteralstrong{\sphinxupquote{phoneControl}} \textendash{} PhoneControl object containing the active connection.

\item[{Returns}] \leavevmode
None.

\end{description}\end{quote}

\end{fulllineitems}

\index{createPhoneConnection() (in module src.Controllers.Program\_Controller)@\spxentry{createPhoneConnection()}\spxextra{in module src.Controllers.Program\_Controller}}

\begin{fulllineitems}
\phantomsection\label{\detokenize{index:src.Controllers.Program_Controller.createPhoneConnection}}\pysiglinewithargsret{\sphinxcode{\sphinxupquote{src.Controllers.Program\_Controller.}}\sphinxbfcode{\sphinxupquote{createPhoneConnection}}}{\emph{portNo}}{{ $\rightarrow$ Controllers.PhoneController.PhoneControl}}
Creates a PhoneControl object used to commmunicate with the phones.
\begin{quote}\begin{description}
\item[{Parameters}] \leavevmode
\sphinxstyleliteralstrong{\sphinxupquote{portNo}} \textendash{} Port number we are going to be listening for signals from the phone over.

\item[{Returns}] \leavevmode
PhoneControl object.

\end{description}\end{quote}

\end{fulllineitems}

\index{main() (in module src.Controllers.Program\_Controller)@\spxentry{main()}\spxextra{in module src.Controllers.Program\_Controller}}

\begin{fulllineitems}
\phantomsection\label{\detokenize{index:src.Controllers.Program_Controller.main}}\pysiglinewithargsret{\sphinxcode{\sphinxupquote{src.Controllers.Program\_Controller.}}\sphinxbfcode{\sphinxupquote{main}}}{}{{ $\rightarrow$ None}}
Begins the main application.
\begin{quote}\begin{description}
\item[{Returns}] \leavevmode
None

\end{description}\end{quote}

\end{fulllineitems}



\chapter{OpenCV}
\label{\detokenize{index:module-src.Controllers.OpenCVThreadedController}}\label{\detokenize{index:opencv}}\index{src.Controllers.OpenCVThreadedController (module)@\spxentry{src.Controllers.OpenCVThreadedController}\spxextra{module}}\index{DroneTracker (class in src.Controllers.OpenCVThreadedController)@\spxentry{DroneTracker}\spxextra{class in src.Controllers.OpenCVThreadedController}}

\begin{fulllineitems}
\phantomsection\label{\detokenize{index:src.Controllers.OpenCVThreadedController.DroneTracker}}\pysiglinewithargsret{\sphinxbfcode{\sphinxupquote{class }}\sphinxcode{\sphinxupquote{src.Controllers.OpenCVThreadedController.}}\sphinxbfcode{\sphinxupquote{DroneTracker}}}{\emph{videoFile}}{}
This class is intended to track sUASs in video recorded from the smartphone app
All containing code to track the drone and output the coordinates extracted from
the recorded video is contained within this class, and this file is intended
to be run as a separate process so that both of the recordings can be processed
in parallel (don’t try this on Windows though).
\index{is\_light\_on() (src.Controllers.OpenCVThreadedController.DroneTracker method)@\spxentry{is\_light\_on()}\spxextra{src.Controllers.OpenCVThreadedController.DroneTracker method}}

\begin{fulllineitems}
\phantomsection\label{\detokenize{index:src.Controllers.OpenCVThreadedController.DroneTracker.is_light_on}}\pysiglinewithargsret{\sphinxbfcode{\sphinxupquote{is\_light\_on}}}{\emph{frame}}{{ $\rightarrow$ bool}}
Takes in a video frame and returns the frame at which the light turns on.
\begin{quote}\begin{description}
\item[{Parameters}] \leavevmode
\sphinxstyleliteralstrong{\sphinxupquote{frame}} \textendash{} A single video frame to see if the light is on.

\item[{Returns}] \leavevmode
True if the light is on, false otherwise.

\end{description}\end{quote}

\end{fulllineitems}

\index{read\_video() (src.Controllers.OpenCVThreadedController.DroneTracker method)@\spxentry{read\_video()}\spxextra{src.Controllers.OpenCVThreadedController.DroneTracker method}}

\begin{fulllineitems}
\phantomsection\label{\detokenize{index:src.Controllers.OpenCVThreadedController.DroneTracker.read_video}}\pysiglinewithargsret{\sphinxbfcode{\sphinxupquote{read\_video}}}{}{{ $\rightarrow$ None}}
This function is to be threaded, and its purpose is to read in the video file
all at once to improve performance.

\end{fulllineitems}

\index{rescale\_frame() (src.Controllers.OpenCVThreadedController.DroneTracker method)@\spxentry{rescale\_frame()}\spxextra{src.Controllers.OpenCVThreadedController.DroneTracker method}}

\begin{fulllineitems}
\phantomsection\label{\detokenize{index:src.Controllers.OpenCVThreadedController.DroneTracker.rescale_frame}}\pysiglinewithargsret{\sphinxbfcode{\sphinxupquote{rescale\_frame}}}{\emph{frame}, \emph{percent=50}}{}
Resizes a frame as a percentage of the original frame size
\begin{quote}\begin{description}
\item[{Parameters}] \leavevmode\begin{itemize}
\item {} 
\sphinxstyleliteralstrong{\sphinxupquote{frame}} \textendash{} the frame to be resized

\item {} 
\sphinxstyleliteralstrong{\sphinxupquote{percent}} \textendash{} the percent value the frame needs to be rescaled to

\end{itemize}

\end{description}\end{quote}

\end{fulllineitems}

\index{resize\_bbox() (src.Controllers.OpenCVThreadedController.DroneTracker method)@\spxentry{resize\_bbox()}\spxextra{src.Controllers.OpenCVThreadedController.DroneTracker method}}

\begin{fulllineitems}
\phantomsection\label{\detokenize{index:src.Controllers.OpenCVThreadedController.DroneTracker.resize_bbox}}\pysiglinewithargsret{\sphinxbfcode{\sphinxupquote{resize\_bbox}}}{\emph{bbox: tuple}, \emph{factor=2}}{{ $\rightarrow$ tuple}}
Resizes the bounding box for translating it to the full size video.
In order to be able to see enough of the footage on screen to draw the box around the drone,
the video frame must be resized, so the drawn bounding box must be translated back into the
coordinate system the full size video uses.
For example, if the 4k footage is shrunk by 50\% (to 1080p), the scale factor here must be 2
so the coordinates chosen in the 1080p frame will match up with the actual drone coordinates in the 4k frame.
\begin{quote}\begin{description}
\item[{Parameters}] \leavevmode\begin{itemize}
\item {} 
\sphinxstyleliteralstrong{\sphinxupquote{bbox}} \textendash{} bounding box of selected drone, which is (x, y, box\_width, box\_height)

\item {} 
\sphinxstyleliteralstrong{\sphinxupquote{factor}} \textendash{} the factor by which to scale the bounding box

\end{itemize}

\item[{Returns}] \leavevmode
tuple

\end{description}\end{quote}

\end{fulllineitems}

\index{trackDrone() (src.Controllers.OpenCVThreadedController.DroneTracker method)@\spxentry{trackDrone()}\spxextra{src.Controllers.OpenCVThreadedController.DroneTracker method}}

\begin{fulllineitems}
\phantomsection\label{\detokenize{index:src.Controllers.OpenCVThreadedController.DroneTracker.trackDrone}}\pysiglinewithargsret{\sphinxbfcode{\sphinxupquote{trackDrone}}}{}{{ $\rightarrow$ list}}
Function that contains all code to track the drone, and is to be run
as a thread. Will run much slower if 2 processes running this method are started and run on different
videos at the same time.
\begin{quote}\begin{description}
\item[{Returns}] \leavevmode
List of tuples of the extracted coordinates of the footage, in the format {[}(time, x\_coord, y\_coord, z\_coord){]}.

\end{description}\end{quote}

\end{fulllineitems}


\end{fulllineitems}

\index{VideoCorruptedException@\spxentry{VideoCorruptedException}}

\begin{fulllineitems}
\phantomsection\label{\detokenize{index:src.Controllers.OpenCVThreadedController.VideoCorruptedException}}\pysiglinewithargsret{\sphinxbfcode{\sphinxupquote{exception }}\sphinxcode{\sphinxupquote{src.Controllers.OpenCVThreadedController.}}\sphinxbfcode{\sphinxupquote{VideoCorruptedException}}}{\emph{message: str}}{}
This error is raised if the video being read is corrupted, or if the frames cannot be
successfully extracted from the video files

\end{fulllineitems}

\index{VideoNotPresentException@\spxentry{VideoNotPresentException}}

\begin{fulllineitems}
\phantomsection\label{\detokenize{index:src.Controllers.OpenCVThreadedController.VideoNotPresentException}}\pysiglinewithargsret{\sphinxbfcode{\sphinxupquote{exception }}\sphinxcode{\sphinxupquote{src.Controllers.OpenCVThreadedController.}}\sphinxbfcode{\sphinxupquote{VideoNotPresentException}}}{\emph{message: str}}{}
This error is raised when the video for processing is not there, or if an incorrect path is given

\end{fulllineitems}

\index{get\_phone\_id() (in module src.Controllers.OpenCVThreadedController)@\spxentry{get\_phone\_id()}\spxextra{in module src.Controllers.OpenCVThreadedController}}

\begin{fulllineitems}
\phantomsection\label{\detokenize{index:src.Controllers.OpenCVThreadedController.get_phone_id}}\pysiglinewithargsret{\sphinxcode{\sphinxupquote{src.Controllers.OpenCVThreadedController.}}\sphinxbfcode{\sphinxupquote{get\_phone\_id}}}{\emph{filename: str}}{{ $\rightarrow$ str}}
Gets the phone Id from the end of the file name so we can keep track of the json and lock files.
\begin{quote}\begin{description}
\item[{Parameters}] \leavevmode
\sphinxstyleliteralstrong{\sphinxupquote{filename}} \textendash{} The file name of the video file. Should have “phone-\#.mp4” file names.

\item[{Returns}] \leavevmode
The ID of the phone from the file name

\end{description}\end{quote}

\end{fulllineitems}

\index{main() (in module src.Controllers.OpenCVThreadedController)@\spxentry{main()}\spxextra{in module src.Controllers.OpenCVThreadedController}}

\begin{fulllineitems}
\phantomsection\label{\detokenize{index:src.Controllers.OpenCVThreadedController.main}}\pysiglinewithargsret{\sphinxcode{\sphinxupquote{src.Controllers.OpenCVThreadedController.}}\sphinxbfcode{\sphinxupquote{main}}}{\emph{filename: str}}{{ $\rightarrow$ None}}
Will take the filename passed in and analyze the footage. All coordinates of the drone in the footage
will be output to a json file.
\begin{quote}\begin{description}
\item[{Returns}] \leavevmode
None

\end{description}\end{quote}

\end{fulllineitems}

\index{merge\_data\_points() (in module src.Controllers.OpenCVThreadedController)@\spxentry{merge\_data\_points()}\spxextra{in module src.Controllers.OpenCVThreadedController}}

\begin{fulllineitems}
\phantomsection\label{\detokenize{index:src.Controllers.OpenCVThreadedController.merge_data_points}}\pysiglinewithargsret{\sphinxcode{\sphinxupquote{src.Controllers.OpenCVThreadedController.}}\sphinxbfcode{\sphinxupquote{merge\_data\_points}}}{\emph{phone1Points: list}, \emph{phone2Points: list}}{{ $\rightarrow$ dict}}
Takes the points outputted by the opencv analysis and merges the points together to create
the 3D coordinates needed to output the visual flight path.
\begin{quote}\begin{description}
\item[{Parameters}] \leavevmode\begin{itemize}
\item {} 
\sphinxstyleliteralstrong{\sphinxupquote{phone1Points}} \textendash{} The opencv datapoints created from the main method of this class for the first phone

\item {} 
\sphinxstyleliteralstrong{\sphinxupquote{phone2Points}} \textendash{} The opencv datapoints created from the main method of this class for the second phone

\end{itemize}

\item[{Returns}] \leavevmode
List of tuples of coordinates and time values that represent the flight path of the drone in the format {[}(time, x\_coord, y\_coord, z\_coord){]}

\end{description}\end{quote}

\end{fulllineitems}



\chapter{Exceptions}
\label{\detokenize{index:module-src.Controllers.Exceptions}}\label{\detokenize{index:exceptions}}\index{src.Controllers.Exceptions (module)@\spxentry{src.Controllers.Exceptions}\spxextra{module}}\index{FailedDisconnectException@\spxentry{FailedDisconnectException}}

\begin{fulllineitems}
\phantomsection\label{\detokenize{index:src.Controllers.Exceptions.FailedDisconnectException}}\pysiglinewithargsret{\sphinxbfcode{\sphinxupquote{exception }}\sphinxcode{\sphinxupquote{src.Controllers.Exceptions.}}\sphinxbfcode{\sphinxupquote{FailedDisconnectException}}}{\emph{message: str}}{}
This error is for telling us that something went wrong when we tried to disconnect from the RPI.

\end{fulllineitems}

\index{FailedRPIFlashException@\spxentry{FailedRPIFlashException}}

\begin{fulllineitems}
\phantomsection\label{\detokenize{index:src.Controllers.Exceptions.FailedRPIFlashException}}\pysiglinewithargsret{\sphinxbfcode{\sphinxupquote{exception }}\sphinxcode{\sphinxupquote{src.Controllers.Exceptions.}}\sphinxbfcode{\sphinxupquote{FailedRPIFlashException}}}{\emph{message: str}}{}
This error is for letting us know that the RPI did not flash the light.

\end{fulllineitems}

\index{PhonesNotSyncedException@\spxentry{PhonesNotSyncedException}}

\begin{fulllineitems}
\phantomsection\label{\detokenize{index:src.Controllers.Exceptions.PhonesNotSyncedException}}\pysiglinewithargsret{\sphinxbfcode{\sphinxupquote{exception }}\sphinxcode{\sphinxupquote{src.Controllers.Exceptions.}}\sphinxbfcode{\sphinxupquote{PhonesNotSyncedException}}}{\emph{message: str}}{}
This exception is for when the user tries to perform an operation with the phones without actually
syncing the phones first.

\end{fulllineitems}

\index{RPINotConnectedException@\spxentry{RPINotConnectedException}}

\begin{fulllineitems}
\phantomsection\label{\detokenize{index:src.Controllers.Exceptions.RPINotConnectedException}}\pysiglinewithargsret{\sphinxbfcode{\sphinxupquote{exception }}\sphinxcode{\sphinxupquote{src.Controllers.Exceptions.}}\sphinxbfcode{\sphinxupquote{RPINotConnectedException}}}{\emph{message: str}}{}
This class is for letting us know that we had an issue connecting to the raspberry pi.

\end{fulllineitems}

\index{RecordingNotStartedException@\spxentry{RecordingNotStartedException}}

\begin{fulllineitems}
\phantomsection\label{\detokenize{index:src.Controllers.Exceptions.RecordingNotStartedException}}\pysiglinewithargsret{\sphinxbfcode{\sphinxupquote{exception }}\sphinxcode{\sphinxupquote{src.Controllers.Exceptions.}}\sphinxbfcode{\sphinxupquote{RecordingNotStartedException}}}{\emph{message: str}}{}
This exception class is for alerting the user that the recording has not started and they
are trying to access a function that requires the phone cameras to be rolling.

\end{fulllineitems}

\index{TransferNotStartedException@\spxentry{TransferNotStartedException}}

\begin{fulllineitems}
\phantomsection\label{\detokenize{index:src.Controllers.Exceptions.TransferNotStartedException}}\pysiglinewithargsret{\sphinxbfcode{\sphinxupquote{exception }}\sphinxcode{\sphinxupquote{src.Controllers.Exceptions.}}\sphinxbfcode{\sphinxupquote{TransferNotStartedException}}}{\emph{message: str}}{}
This exception class is for alerting the user that the file transfer process has not been
started and any actions that depend on it will fail.

\end{fulllineitems}



\chapter{PhoneController}
\label{\detokenize{index:module-src.Controllers.PhoneController}}\label{\detokenize{index:phonecontroller}}\index{src.Controllers.PhoneController (module)@\spxentry{src.Controllers.PhoneController}\spxextra{module}}\index{PhoneControl (class in src.Controllers.PhoneController)@\spxentry{PhoneControl}\spxextra{class in src.Controllers.PhoneController}}

\begin{fulllineitems}
\phantomsection\label{\detokenize{index:src.Controllers.PhoneController.PhoneControl}}\pysiglinewithargsret{\sphinxbfcode{\sphinxupquote{class }}\sphinxcode{\sphinxupquote{src.Controllers.PhoneController.}}\sphinxbfcode{\sphinxupquote{PhoneControl}}}{\emph{portNum: int}}{}
This class is for communicating with and controlling the phones out in the field. It uses simple
TCP connections with each phone in order to control them.
\index{closeConn() (src.Controllers.PhoneController.PhoneControl method)@\spxentry{closeConn()}\spxextra{src.Controllers.PhoneController.PhoneControl method}}

\begin{fulllineitems}
\phantomsection\label{\detokenize{index:src.Controllers.PhoneController.PhoneControl.closeConn}}\pysiglinewithargsret{\sphinxbfcode{\sphinxupquote{closeConn}}}{}{{ $\rightarrow$ None}}
Closes all of the connections and the socket.
:return: None

\end{fulllineitems}

\index{isTransferring() (src.Controllers.PhoneController.PhoneControl method)@\spxentry{isTransferring()}\spxextra{src.Controllers.PhoneController.PhoneControl method}}

\begin{fulllineitems}
\phantomsection\label{\detokenize{index:src.Controllers.PhoneController.PhoneControl.isTransferring}}\pysiglinewithargsret{\sphinxbfcode{\sphinxupquote{isTransferring}}}{}{{ $\rightarrow$ bool}}
A flag for us to access to see if the system is still waiting for the video to finish the file
transfer of the videos it recorded.
\begin{quote}\begin{description}
\item[{Returns}] \leavevmode
True if the video has finished transferring, false otherwise.

\end{description}\end{quote}

\end{fulllineitems}

\index{setupSocket() (src.Controllers.PhoneController.PhoneControl method)@\spxentry{setupSocket()}\spxextra{src.Controllers.PhoneController.PhoneControl method}}

\begin{fulllineitems}
\phantomsection\label{\detokenize{index:src.Controllers.PhoneController.PhoneControl.setupSocket}}\pysiglinewithargsret{\sphinxbfcode{\sphinxupquote{setupSocket}}}{}{{ $\rightarrow$ None}}
Creates the socket that we will use to listen for incoming connections.
\begin{quote}\begin{description}
\item[{Returns}] \leavevmode
None

\end{description}\end{quote}

\end{fulllineitems}

\index{startFileTransfer() (src.Controllers.PhoneController.PhoneControl method)@\spxentry{startFileTransfer()}\spxextra{src.Controllers.PhoneController.PhoneControl method}}

\begin{fulllineitems}
\phantomsection\label{\detokenize{index:src.Controllers.PhoneController.PhoneControl.startFileTransfer}}\pysiglinewithargsret{\sphinxbfcode{\sphinxupquote{startFileTransfer}}}{\emph{filepath: str}}{{ $\rightarrow$ None}}
Will send a signal to the phone that tells it to transfer the video files it recorded over to the laptop
by opening an FTP connection.
\begin{quote}\begin{description}
\item[{Parameters}] \leavevmode
\sphinxstyleliteralstrong{\sphinxupquote{filepath}} \textendash{} The file path on our laptop that the phones will need to send their videos to over FTP.

\item[{Returns}] \leavevmode
None

\end{description}\end{quote}

\end{fulllineitems}

\index{startRecording() (src.Controllers.PhoneController.PhoneControl method)@\spxentry{startRecording()}\spxextra{src.Controllers.PhoneController.PhoneControl method}}

\begin{fulllineitems}
\phantomsection\label{\detokenize{index:src.Controllers.PhoneController.PhoneControl.startRecording}}\pysiglinewithargsret{\sphinxbfcode{\sphinxupquote{startRecording}}}{}{{ $\rightarrow$ None}}
Call this in order to send a signal to the phones that they need to start recording.
\begin{quote}\begin{description}
\item[{Returns}] \leavevmode
None

\end{description}\end{quote}

\end{fulllineitems}

\index{stopRecording() (src.Controllers.PhoneController.PhoneControl method)@\spxentry{stopRecording()}\spxextra{src.Controllers.PhoneController.PhoneControl method}}

\begin{fulllineitems}
\phantomsection\label{\detokenize{index:src.Controllers.PhoneController.PhoneControl.stopRecording}}\pysiglinewithargsret{\sphinxbfcode{\sphinxupquote{stopRecording}}}{}{{ $\rightarrow$ None}}
Call this in order to send a signal to the phones that they need to stop tracking. Sends both
phones a stop signal and the name of the file path that they will need to send their
videos to over FTP.
\begin{quote}\begin{description}
\item[{Returns}] \leavevmode
None

\end{description}\end{quote}

\end{fulllineitems}

\index{sync() (src.Controllers.PhoneController.PhoneControl method)@\spxentry{sync()}\spxextra{src.Controllers.PhoneController.PhoneControl method}}

\begin{fulllineitems}
\phantomsection\label{\detokenize{index:src.Controllers.PhoneController.PhoneControl.sync}}\pysiglinewithargsret{\sphinxbfcode{\sphinxupquote{sync}}}{}{{ $\rightarrow$ None}}
This function will wait until both phones have been connected to this app.
\begin{quote}\begin{description}
\item[{Returns}] \leavevmode
None

\end{description}\end{quote}

\end{fulllineitems}

\index{synced() (src.Controllers.PhoneController.PhoneControl method)@\spxentry{synced()}\spxextra{src.Controllers.PhoneController.PhoneControl method}}

\begin{fulllineitems}
\phantomsection\label{\detokenize{index:src.Controllers.PhoneController.PhoneControl.synced}}\pysiglinewithargsret{\sphinxbfcode{\sphinxupquote{synced}}}{}{{ $\rightarrow$ bool}}
The getter function for seeing if the phones synced or not.
\begin{quote}\begin{description}
\item[{Returns}] \leavevmode
True if they have been synced, false otherwise

\end{description}\end{quote}

\end{fulllineitems}

\index{threadSendSignal() (src.Controllers.PhoneController.PhoneControl method)@\spxentry{threadSendSignal()}\spxextra{src.Controllers.PhoneController.PhoneControl method}}

\begin{fulllineitems}
\phantomsection\label{\detokenize{index:src.Controllers.PhoneController.PhoneControl.threadSendSignal}}\pysiglinewithargsret{\sphinxbfcode{\sphinxupquote{threadSendSignal}}}{\emph{conn: socket.socket}, \emph{signal: str}, \emph{sigMessage: str}, \emph{sigAck: str}}{{ $\rightarrow$ None}}
This function takes a signal and the expected output so that we don’t have to rewrite the same code for
every action we have with the phones.
\begin{quote}\begin{description}
\item[{Parameters}] \leavevmode\begin{itemize}
\item {} 
\sphinxstyleliteralstrong{\sphinxupquote{conn}} \textendash{} A socket connection to a phone that has already been opened.

\item {} 
\sphinxstyleliteralstrong{\sphinxupquote{signal}} \textendash{} The Signal we want to send to the phone. Valid options are: START, STOP, and START\_FTP

\item {} 
\sphinxstyleliteralstrong{\sphinxupquote{sigMessage}} \textendash{} A message that we want to send alongside the signal for the phone to use.

\item {} 
\sphinxstyleliteralstrong{\sphinxupquote{sigAck}} \textendash{} The Signal we expect to get back from the phone in response to our signal. Valid options are: START\_ACKNOWLEDGE, STOP\_ACKNOWLEDGE, START\_FTP\_ACKNOWLEDGE.

\end{itemize}

\item[{Returns}] \leavevmode
None

\end{description}\end{quote}

\end{fulllineitems}

\index{threadWaitForFileTransfer() (src.Controllers.PhoneController.PhoneControl method)@\spxentry{threadWaitForFileTransfer()}\spxextra{src.Controllers.PhoneController.PhoneControl method}}

\begin{fulllineitems}
\phantomsection\label{\detokenize{index:src.Controllers.PhoneController.PhoneControl.threadWaitForFileTransfer}}\pysiglinewithargsret{\sphinxbfcode{\sphinxupquote{threadWaitForFileTransfer}}}{\emph{conn: socket.socket}}{{ $\rightarrow$ None}}
This is a thread for waiting for the signal from the phone that the file transfer to the filepath specified
in the startFileTransfer() function arguments.
\begin{quote}\begin{description}
\item[{Parameters}] \leavevmode
\sphinxstyleliteralstrong{\sphinxupquote{conn}} \textendash{} A socket connection to a phone that has already been opened.

\item[{Returns}] \leavevmode
None

\end{description}\end{quote}

\end{fulllineitems}

\index{waitForFileTransfer() (src.Controllers.PhoneController.PhoneControl method)@\spxentry{waitForFileTransfer()}\spxextra{src.Controllers.PhoneController.PhoneControl method}}

\begin{fulllineitems}
\phantomsection\label{\detokenize{index:src.Controllers.PhoneController.PhoneControl.waitForFileTransfer}}\pysiglinewithargsret{\sphinxbfcode{\sphinxupquote{waitForFileTransfer}}}{}{{ $\rightarrow$ None}}
Spawns threads that will wait for both phones to send a signal saying that the file transfer
of the videos is complete.
\begin{quote}\begin{description}
\item[{Returns}] \leavevmode
None

\end{description}\end{quote}

\end{fulllineitems}


\end{fulllineitems}



\chapter{Export Flight}
\label{\detokenize{index:module-src.Export.ExportFile}}\label{\detokenize{index:export-flight}}\index{src.Export.ExportFile (module)@\spxentry{src.Export.ExportFile}\spxextra{module}}\index{export\_data() (in module src.Export.ExportFile)@\spxentry{export\_data()}\spxextra{in module src.Export.ExportFile}}

\begin{fulllineitems}
\phantomsection\label{\detokenize{index:src.Export.ExportFile.export_data}}\pysiglinewithargsret{\sphinxcode{\sphinxupquote{src.Export.ExportFile.}}\sphinxbfcode{\sphinxupquote{export\_data}}}{\emph{pilotName}, \emph{instructorName}, \emph{flightDate}, \emph{flightLength}, \emph{flightInstructions}, \emph{xCoordinates}, \emph{yCoordinates}, \emph{zCoordinates}, \emph{velocityValues}, \emph{outPath}}{{ $\rightarrow$ None}}
Exports the flight data to a JSON file stored with a “.flight” extension.
\begin{quote}\begin{description}
\item[{Parameters}] \leavevmode\begin{itemize}
\item {} 
\sphinxstyleliteralstrong{\sphinxupquote{pilotName}} \textendash{} String containing the pilot name

\item {} 
\sphinxstyleliteralstrong{\sphinxupquote{instructorName}} \textendash{} String containing the instructor name

\item {} 
\sphinxstyleliteralstrong{\sphinxupquote{flightDate}} \textendash{} String containing the flight date

\item {} 
\sphinxstyleliteralstrong{\sphinxupquote{flightLength}} \textendash{} String containing the flight length

\item {} 
\sphinxstyleliteralstrong{\sphinxupquote{flightInstructions}} \textendash{} String containing the flight instructions

\item {} 
\sphinxstyleliteralstrong{\sphinxupquote{xCoordinates}} \textendash{} Array of x coordinates

\item {} 
\sphinxstyleliteralstrong{\sphinxupquote{yCoordinates}} \textendash{} Array of y coordinates

\item {} 
\sphinxstyleliteralstrong{\sphinxupquote{zCoordinates}} \textendash{} Array of z coordinates

\item {} 
\sphinxstyleliteralstrong{\sphinxupquote{velocityValues}} \textendash{} Array of velocity values

\item {} 
\sphinxstyleliteralstrong{\sphinxupquote{outPath}} \textendash{} String containing the path to save the file. Should end in “.flight”.

\end{itemize}

\item[{Returns}] \leavevmode
none

\end{description}\end{quote}

\end{fulllineitems}



\chapter{Import Flight}
\label{\detokenize{index:module-src.Export.ImportFile}}\label{\detokenize{index:import-flight}}\index{src.Export.ImportFile (module)@\spxentry{src.Export.ImportFile}\spxextra{module}}\index{importData() (in module src.Export.ImportFile)@\spxentry{importData()}\spxextra{in module src.Export.ImportFile}}

\begin{fulllineitems}
\phantomsection\label{\detokenize{index:src.Export.ImportFile.importData}}\pysiglinewithargsret{\sphinxcode{\sphinxupquote{src.Export.ImportFile.}}\sphinxbfcode{\sphinxupquote{importData}}}{\emph{inPath}}{{ $\rightarrow$ dict}}
Imports the flight data from a JSON file stored with a “.flight” extension.
\begin{quote}\begin{description}
\item[{Parameters}] \leavevmode
\sphinxstyleliteralstrong{\sphinxupquote{inPath}} \textendash{} String containing the pilot name.

\item[{Returns}] \leavevmode
Flight dictionary

\end{description}\end{quote}

\end{fulllineitems}



\chapter{Loading Screen}
\label{\detokenize{index:module-src.Views.View_LoadingScreen}}\label{\detokenize{index:loading-screen}}\index{src.Views.View\_LoadingScreen (module)@\spxentry{src.Views.View\_LoadingScreen}\spxextra{module}}\index{LoadingWindow (class in src.Views.View\_LoadingScreen)@\spxentry{LoadingWindow}\spxextra{class in src.Views.View\_LoadingScreen}}

\begin{fulllineitems}
\phantomsection\label{\detokenize{index:src.Views.View_LoadingScreen.LoadingWindow}}\pysigline{\sphinxbfcode{\sphinxupquote{class }}\sphinxcode{\sphinxupquote{src.Views.View\_LoadingScreen.}}\sphinxbfcode{\sphinxupquote{LoadingWindow}}}
The view for the loading page that is shown when the user presses the “Stop Tracking” button on the tracking window page.
\begin{quote}\begin{description}
\item[{Variables}] \leavevmode
\sphinxstyleliteralstrong{\sphinxupquote{\_\_btnHome}} \textendash{} The class property for the ‘Return to Home’ button.

\end{description}\end{quote}
\index{BtnHome() (src.Views.View\_LoadingScreen.LoadingWindow property)@\spxentry{BtnHome()}\spxextra{src.Views.View\_LoadingScreen.LoadingWindow property}}

\begin{fulllineitems}
\phantomsection\label{\detokenize{index:src.Views.View_LoadingScreen.LoadingWindow.BtnHome}}\pysigline{\sphinxbfcode{\sphinxupquote{property }}\sphinxbfcode{\sphinxupquote{BtnHome}}}
The home for the view. Is used to return to home screen.
\begin{quote}\begin{description}
\item[{Returns}] \leavevmode
None

\end{description}\end{quote}

\end{fulllineitems}

\index{BtnTestReport() (src.Views.View\_LoadingScreen.LoadingWindow property)@\spxentry{BtnTestReport()}\spxextra{src.Views.View\_LoadingScreen.LoadingWindow property}}

\begin{fulllineitems}
\phantomsection\label{\detokenize{index:src.Views.View_LoadingScreen.LoadingWindow.BtnTestReport}}\pysigline{\sphinxbfcode{\sphinxupquote{property }}\sphinxbfcode{\sphinxupquote{BtnTestReport}}}
The test report for the view. Is used to switch to the test report screen.
\begin{quote}\begin{description}
\item[{Returns}] \leavevmode
The reference to the test report button.

\end{description}\end{quote}

\end{fulllineitems}

\index{LblStatus() (src.Views.View\_LoadingScreen.LoadingWindow property)@\spxentry{LblStatus()}\spxextra{src.Views.View\_LoadingScreen.LoadingWindow property}}

\begin{fulllineitems}
\phantomsection\label{\detokenize{index:src.Views.View_LoadingScreen.LoadingWindow.LblStatus}}\pysigline{\sphinxbfcode{\sphinxupquote{property }}\sphinxbfcode{\sphinxupquote{LblStatus}}}
Getter property for the timer label. We need to attach a QTimer to it so it can count the time the
application has been tracking the drone.
\begin{quote}\begin{description}
\item[{Returns}] \leavevmode
The timer label

\end{description}\end{quote}

\end{fulllineitems}

\index{del\_BtnHome() (src.Views.View\_LoadingScreen.LoadingWindow property)@\spxentry{del\_BtnHome()}\spxextra{src.Views.View\_LoadingScreen.LoadingWindow property}}

\begin{fulllineitems}
\phantomsection\label{\detokenize{index:src.Views.View_LoadingScreen.LoadingWindow.del_BtnHome}}\pysigline{\sphinxbfcode{\sphinxupquote{property }}\sphinxbfcode{\sphinxupquote{del\_BtnHome}}}
The home for the view. Is used to return to home screen.
\begin{quote}\begin{description}
\item[{Returns}] \leavevmode
None

\end{description}\end{quote}

\end{fulllineitems}

\index{del\_BtnTestReport() (src.Views.View\_LoadingScreen.LoadingWindow property)@\spxentry{del\_BtnTestReport()}\spxextra{src.Views.View\_LoadingScreen.LoadingWindow property}}

\begin{fulllineitems}
\phantomsection\label{\detokenize{index:src.Views.View_LoadingScreen.LoadingWindow.del_BtnTestReport}}\pysigline{\sphinxbfcode{\sphinxupquote{property }}\sphinxbfcode{\sphinxupquote{del\_BtnTestReport}}}
The test report for the view. Is used to switch to the test report screen.
\begin{quote}\begin{description}
\item[{Returns}] \leavevmode
The reference to the test report button.

\end{description}\end{quote}

\end{fulllineitems}

\index{del\_LblStatus() (src.Views.View\_LoadingScreen.LoadingWindow property)@\spxentry{del\_LblStatus()}\spxextra{src.Views.View\_LoadingScreen.LoadingWindow property}}

\begin{fulllineitems}
\phantomsection\label{\detokenize{index:src.Views.View_LoadingScreen.LoadingWindow.del_LblStatus}}\pysigline{\sphinxbfcode{\sphinxupquote{property }}\sphinxbfcode{\sphinxupquote{del\_LblStatus}}}
Getter property for the timer label. We need to attach a QTimer to it so it can count the time the
application has been tracking the drone.
\begin{quote}\begin{description}
\item[{Returns}] \leavevmode
The timer label

\end{description}\end{quote}

\end{fulllineitems}

\index{initView() (src.Views.View\_LoadingScreen.LoadingWindow method)@\spxentry{initView()}\spxextra{src.Views.View\_LoadingScreen.LoadingWindow method}}

\begin{fulllineitems}
\phantomsection\label{\detokenize{index:src.Views.View_LoadingScreen.LoadingWindow.initView}}\pysiglinewithargsret{\sphinxbfcode{\sphinxupquote{initView}}}{}{{ $\rightarrow$ None}}
Sets up the view and lays out all of the components.
\begin{quote}\begin{description}
\item[{Returns}] \leavevmode
None

\end{description}\end{quote}

\end{fulllineitems}

\index{returnHome() (src.Views.View\_LoadingScreen.LoadingWindow method)@\spxentry{returnHome()}\spxextra{src.Views.View\_LoadingScreen.LoadingWindow method}}

\begin{fulllineitems}
\phantomsection\label{\detokenize{index:src.Views.View_LoadingScreen.LoadingWindow.returnHome}}\pysiglinewithargsret{\sphinxbfcode{\sphinxupquote{returnHome}}}{}{{ $\rightarrow$ None}}
Sends a signal to the main controller that the Cancel and Return to Home button was pushed.
\begin{quote}\begin{description}
\item[{Returns}] \leavevmode
none

\end{description}\end{quote}

\end{fulllineitems}

\index{setSubtitle() (src.Views.View\_LoadingScreen.LoadingWindow method)@\spxentry{setSubtitle()}\spxextra{src.Views.View\_LoadingScreen.LoadingWindow method}}

\begin{fulllineitems}
\phantomsection\label{\detokenize{index:src.Views.View_LoadingScreen.LoadingWindow.setSubtitle}}\pysiglinewithargsret{\sphinxbfcode{\sphinxupquote{setSubtitle}}}{}{{ $\rightarrow$ PyQt5.QtWidgets.QLabel}}
Sets up the subtitle label.
\begin{quote}\begin{description}
\item[{Returns}] \leavevmode
The subtitle label

\end{description}\end{quote}

\end{fulllineitems}

\index{setTitle() (src.Views.View\_LoadingScreen.LoadingWindow method)@\spxentry{setTitle()}\spxextra{src.Views.View\_LoadingScreen.LoadingWindow method}}

\begin{fulllineitems}
\phantomsection\label{\detokenize{index:src.Views.View_LoadingScreen.LoadingWindow.setTitle}}\pysiglinewithargsret{\sphinxbfcode{\sphinxupquote{setTitle}}}{}{{ $\rightarrow$ PyQt5.QtWidgets.QLabel}}
Sets up the title with the application title on top and the name of the screen just below it.
\begin{quote}\begin{description}
\item[{Returns}] \leavevmode
Layout with the application title and screen title labels

\end{description}\end{quote}

\end{fulllineitems}

\index{set\_BtnHome() (src.Views.View\_LoadingScreen.LoadingWindow property)@\spxentry{set\_BtnHome()}\spxextra{src.Views.View\_LoadingScreen.LoadingWindow property}}

\begin{fulllineitems}
\phantomsection\label{\detokenize{index:src.Views.View_LoadingScreen.LoadingWindow.set_BtnHome}}\pysigline{\sphinxbfcode{\sphinxupquote{property }}\sphinxbfcode{\sphinxupquote{set\_BtnHome}}}
The home for the view. Is used to return to home screen.
\begin{quote}\begin{description}
\item[{Returns}] \leavevmode
None

\end{description}\end{quote}

\end{fulllineitems}

\index{set\_BtnTestReport() (src.Views.View\_LoadingScreen.LoadingWindow property)@\spxentry{set\_BtnTestReport()}\spxextra{src.Views.View\_LoadingScreen.LoadingWindow property}}

\begin{fulllineitems}
\phantomsection\label{\detokenize{index:src.Views.View_LoadingScreen.LoadingWindow.set_BtnTestReport}}\pysigline{\sphinxbfcode{\sphinxupquote{property }}\sphinxbfcode{\sphinxupquote{set\_BtnTestReport}}}
The test report for the view. Is used to switch to the test report screen.
\begin{quote}\begin{description}
\item[{Returns}] \leavevmode
The reference to the test report button.

\end{description}\end{quote}

\end{fulllineitems}

\index{set\_LblStatus() (src.Views.View\_LoadingScreen.LoadingWindow property)@\spxentry{set\_LblStatus()}\spxextra{src.Views.View\_LoadingScreen.LoadingWindow property}}

\begin{fulllineitems}
\phantomsection\label{\detokenize{index:src.Views.View_LoadingScreen.LoadingWindow.set_LblStatus}}\pysigline{\sphinxbfcode{\sphinxupquote{property }}\sphinxbfcode{\sphinxupquote{set\_LblStatus}}}
Getter property for the timer label. We need to attach a QTimer to it so it can count the time the
application has been tracking the drone.
\begin{quote}\begin{description}
\item[{Returns}] \leavevmode
The timer label

\end{description}\end{quote}

\end{fulllineitems}

\index{setupLoadingIcon() (src.Views.View\_LoadingScreen.LoadingWindow method)@\spxentry{setupLoadingIcon()}\spxextra{src.Views.View\_LoadingScreen.LoadingWindow method}}

\begin{fulllineitems}
\phantomsection\label{\detokenize{index:src.Views.View_LoadingScreen.LoadingWindow.setupLoadingIcon}}\pysiglinewithargsret{\sphinxbfcode{\sphinxupquote{setupLoadingIcon}}}{}{{ $\rightarrow$ PyQt5.QtWidgets.QLabel}}
Used for configuring the loading icon on the loading screen.
Loading icon is a gif, so QMovie is used to animate the icon.
\begin{quote}\begin{description}
\item[{Returns}] \leavevmode
The icon containing the loading label.

\end{description}\end{quote}

\end{fulllineitems}

\index{signalTestReport() (src.Views.View\_LoadingScreen.LoadingWindow method)@\spxentry{signalTestReport()}\spxextra{src.Views.View\_LoadingScreen.LoadingWindow method}}

\begin{fulllineitems}
\phantomsection\label{\detokenize{index:src.Views.View_LoadingScreen.LoadingWindow.signalTestReport}}\pysiglinewithargsret{\sphinxbfcode{\sphinxupquote{signalTestReport}}}{}{{ $\rightarrow$ None}}
Sends a signal to the main controller that the Test Report button was pushed.
NOTE: ONLY USED FOR TESTING PURPOSES
\begin{quote}\begin{description}
\item[{Returns}] \leavevmode
none

\end{description}\end{quote}

\end{fulllineitems}

\index{signalTransferFootage() (src.Views.View\_LoadingScreen.LoadingWindow method)@\spxentry{signalTransferFootage()}\spxextra{src.Views.View\_LoadingScreen.LoadingWindow method}}

\begin{fulllineitems}
\phantomsection\label{\detokenize{index:src.Views.View_LoadingScreen.LoadingWindow.signalTransferFootage}}\pysiglinewithargsret{\sphinxbfcode{\sphinxupquote{signalTransferFootage}}}{}{{ $\rightarrow$ None}}
Sends a signal to the main controller that the button to transfer footage was pressed.
\begin{quote}\begin{description}
\item[{Returns}] \leavevmode
none

\end{description}\end{quote}

\end{fulllineitems}


\end{fulllineitems}



\chapter{Report Screen}
\label{\detokenize{index:module-src.Views.View_ReportScreen}}\label{\detokenize{index:report-screen}}\index{src.Views.View\_ReportScreen (module)@\spxentry{src.Views.View\_ReportScreen}\spxextra{module}}\index{ReportWindow (class in src.Views.View\_ReportScreen)@\spxentry{ReportWindow}\spxextra{class in src.Views.View\_ReportScreen}}

\begin{fulllineitems}
\phantomsection\label{\detokenize{index:src.Views.View_ReportScreen.ReportWindow}}\pysiglinewithargsret{\sphinxbfcode{\sphinxupquote{class }}\sphinxcode{\sphinxupquote{src.Views.View\_ReportScreen.}}\sphinxbfcode{\sphinxupquote{ReportWindow}}}{\emph{pilotName: str}, \emph{instructorName: str}, \emph{flightInstructions: str}, \emph{previousFlight: str}, \emph{usingPreviousFlight: bool}, \emph{flightData: dict}}{}
The view for the report page that is shown when the user opens the application.
\begin{quote}\begin{description}
\item[{Variables}] \leavevmode\begin{itemize}
\item {} 
\sphinxstyleliteralstrong{\sphinxupquote{\_\_btnExport}} \textendash{} The class property for the ‘Export Results’ button.

\item {} 
\sphinxstyleliteralstrong{\sphinxupquote{\_\_btnFlyAgain}} \textendash{} The class property for the ‘Fly Again’ button.

\item {} 
\sphinxstyleliteralstrong{\sphinxupquote{\_\_btnHome}} \textendash{} The class property for the ‘Return to Home’ button.

\item {} 
\sphinxstyleliteralstrong{\sphinxupquote{\_\_btnViewGraphVelocity}} \textendash{} The class property for the ‘View Flight Path’ button.

\item {} 
\sphinxstyleliteralstrong{\sphinxupquote{\_\_btnViewGraphNoVelocity}} \textendash{} The class property for the ‘View Flight Path with Velocity Changes’ button.

\item {} 
\sphinxstyleliteralstrong{\sphinxupquote{\_\_btnViewInstructions}} \textendash{} The class property for the ‘View Flight Instructions’ button.

\end{itemize}

\end{description}\end{quote}
\index{BtnExport() (src.Views.View\_ReportScreen.ReportWindow property)@\spxentry{BtnExport()}\spxextra{src.Views.View\_ReportScreen.ReportWindow property}}

\begin{fulllineitems}
\phantomsection\label{\detokenize{index:src.Views.View_ReportScreen.ReportWindow.BtnExport}}\pysigline{\sphinxbfcode{\sphinxupquote{property }}\sphinxbfcode{\sphinxupquote{BtnExport}}}
The export button for the view.
\begin{quote}\begin{description}
\item[{Returns}] \leavevmode
None

\end{description}\end{quote}

\end{fulllineitems}

\index{BtnFlyAgain() (src.Views.View\_ReportScreen.ReportWindow property)@\spxentry{BtnFlyAgain()}\spxextra{src.Views.View\_ReportScreen.ReportWindow property}}

\begin{fulllineitems}
\phantomsection\label{\detokenize{index:src.Views.View_ReportScreen.ReportWindow.BtnFlyAgain}}\pysigline{\sphinxbfcode{\sphinxupquote{property }}\sphinxbfcode{\sphinxupquote{BtnFlyAgain}}}
The fly again button for the view.
\begin{quote}\begin{description}
\item[{Returns}] \leavevmode
None

\end{description}\end{quote}

\end{fulllineitems}

\index{BtnHome() (src.Views.View\_ReportScreen.ReportWindow property)@\spxentry{BtnHome()}\spxextra{src.Views.View\_ReportScreen.ReportWindow property}}

\begin{fulllineitems}
\phantomsection\label{\detokenize{index:src.Views.View_ReportScreen.ReportWindow.BtnHome}}\pysigline{\sphinxbfcode{\sphinxupquote{property }}\sphinxbfcode{\sphinxupquote{BtnHome}}}
The home for the view. Is used to return to home screen.
\begin{quote}\begin{description}
\item[{Returns}] \leavevmode
None

\end{description}\end{quote}

\end{fulllineitems}

\index{BtnViewGraphNoVelocity() (src.Views.View\_ReportScreen.ReportWindow property)@\spxentry{BtnViewGraphNoVelocity()}\spxextra{src.Views.View\_ReportScreen.ReportWindow property}}

\begin{fulllineitems}
\phantomsection\label{\detokenize{index:src.Views.View_ReportScreen.ReportWindow.BtnViewGraphNoVelocity}}\pysigline{\sphinxbfcode{\sphinxupquote{property }}\sphinxbfcode{\sphinxupquote{BtnViewGraphNoVelocity}}}
The home for the view graph without velocity button.
\begin{quote}\begin{description}
\item[{Returns}] \leavevmode
None

\end{description}\end{quote}

\end{fulllineitems}

\index{BtnViewGraphVelocity() (src.Views.View\_ReportScreen.ReportWindow property)@\spxentry{BtnViewGraphVelocity()}\spxextra{src.Views.View\_ReportScreen.ReportWindow property}}

\begin{fulllineitems}
\phantomsection\label{\detokenize{index:src.Views.View_ReportScreen.ReportWindow.BtnViewGraphVelocity}}\pysigline{\sphinxbfcode{\sphinxupquote{property }}\sphinxbfcode{\sphinxupquote{BtnViewGraphVelocity}}}
The home for the view graph with velocity button.
\begin{quote}\begin{description}
\item[{Returns}] \leavevmode
None

\end{description}\end{quote}

\end{fulllineitems}

\index{LblFlightDate() (src.Views.View\_ReportScreen.ReportWindow property)@\spxentry{LblFlightDate()}\spxextra{src.Views.View\_ReportScreen.ReportWindow property}}

\begin{fulllineitems}
\phantomsection\label{\detokenize{index:src.Views.View_ReportScreen.ReportWindow.LblFlightDate}}\pysigline{\sphinxbfcode{\sphinxupquote{property }}\sphinxbfcode{\sphinxupquote{LblFlightDate}}}
Getter for the flight date label.
\begin{quote}\begin{description}
\item[{Returns}] \leavevmode
Reference to the flight date label.

\end{description}\end{quote}

\end{fulllineitems}

\index{LblFlightInstructions() (src.Views.View\_ReportScreen.ReportWindow property)@\spxentry{LblFlightInstructions()}\spxextra{src.Views.View\_ReportScreen.ReportWindow property}}

\begin{fulllineitems}
\phantomsection\label{\detokenize{index:src.Views.View_ReportScreen.ReportWindow.LblFlightInstructions}}\pysigline{\sphinxbfcode{\sphinxupquote{property }}\sphinxbfcode{\sphinxupquote{LblFlightInstructions}}}
Getter for the flight instructions label.
\begin{quote}\begin{description}
\item[{Returns}] \leavevmode
Reference to the flight length label.

\end{description}\end{quote}

\end{fulllineitems}

\index{LblFlightLength() (src.Views.View\_ReportScreen.ReportWindow property)@\spxentry{LblFlightLength()}\spxextra{src.Views.View\_ReportScreen.ReportWindow property}}

\begin{fulllineitems}
\phantomsection\label{\detokenize{index:src.Views.View_ReportScreen.ReportWindow.LblFlightLength}}\pysigline{\sphinxbfcode{\sphinxupquote{property }}\sphinxbfcode{\sphinxupquote{LblFlightLength}}}
Getter for the flight length label.
\begin{quote}\begin{description}
\item[{Returns}] \leavevmode
Reference to the flight length label.

\end{description}\end{quote}

\end{fulllineitems}

\index{LblInstructor() (src.Views.View\_ReportScreen.ReportWindow property)@\spxentry{LblInstructor()}\spxextra{src.Views.View\_ReportScreen.ReportWindow property}}

\begin{fulllineitems}
\phantomsection\label{\detokenize{index:src.Views.View_ReportScreen.ReportWindow.LblInstructor}}\pysigline{\sphinxbfcode{\sphinxupquote{property }}\sphinxbfcode{\sphinxupquote{LblInstructor}}}
Getter for the instructor label.
\begin{quote}\begin{description}
\item[{Returns}] \leavevmode
Reference to the instructor label.

\end{description}\end{quote}

\end{fulllineitems}

\index{LblPilot() (src.Views.View\_ReportScreen.ReportWindow property)@\spxentry{LblPilot()}\spxextra{src.Views.View\_ReportScreen.ReportWindow property}}

\begin{fulllineitems}
\phantomsection\label{\detokenize{index:src.Views.View_ReportScreen.ReportWindow.LblPilot}}\pysigline{\sphinxbfcode{\sphinxupquote{property }}\sphinxbfcode{\sphinxupquote{LblPilot}}}
Getter for the Pilot label so we can set who the pilot is for the flight in child class.
\begin{quote}\begin{description}
\item[{Returns}] \leavevmode
Reference to the pilot label.

\end{description}\end{quote}

\end{fulllineitems}

\index{analyzeFlight() (src.Views.View\_ReportScreen.ReportWindow method)@\spxentry{analyzeFlight()}\spxextra{src.Views.View\_ReportScreen.ReportWindow method}}

\begin{fulllineitems}
\phantomsection\label{\detokenize{index:src.Views.View_ReportScreen.ReportWindow.analyzeFlight}}\pysiglinewithargsret{\sphinxbfcode{\sphinxupquote{analyzeFlight}}}{\emph{flightDict: dict}}{{ $\rightarrow$ dict}}
Analyzes the flight data to extract coordinates, velocity values, and statistics.
\begin{quote}\begin{description}
\item[{Parameters}] \leavevmode
\sphinxstyleliteralstrong{\sphinxupquote{flightDict}} \textendash{} Dictionary of flight data, with only coordinates populated.

\item[{Returns}] \leavevmode
Updated dictionary, with legal points and flight statistics included.

\end{description}\end{quote}

\end{fulllineitems}

\index{createStatisticsTable() (src.Views.View\_ReportScreen.ReportWindow method)@\spxentry{createStatisticsTable()}\spxextra{src.Views.View\_ReportScreen.ReportWindow method}}

\begin{fulllineitems}
\phantomsection\label{\detokenize{index:src.Views.View_ReportScreen.ReportWindow.createStatisticsTable}}\pysiglinewithargsret{\sphinxbfcode{\sphinxupquote{createStatisticsTable}}}{}{{ $\rightarrow$ PyQt5.QtWidgets.QTableWidget}}
Creates a table containing flight statistics.
\begin{quote}\begin{description}
\item[{Returns}] \leavevmode
QTableWidget containing flight statistics.

\end{description}\end{quote}

\end{fulllineitems}

\index{del\_BtnExport() (src.Views.View\_ReportScreen.ReportWindow property)@\spxentry{del\_BtnExport()}\spxextra{src.Views.View\_ReportScreen.ReportWindow property}}

\begin{fulllineitems}
\phantomsection\label{\detokenize{index:src.Views.View_ReportScreen.ReportWindow.del_BtnExport}}\pysigline{\sphinxbfcode{\sphinxupquote{property }}\sphinxbfcode{\sphinxupquote{del\_BtnExport}}}
The export button for the view.
\begin{quote}\begin{description}
\item[{Returns}] \leavevmode
None

\end{description}\end{quote}

\end{fulllineitems}

\index{del\_BtnFlyAgain() (src.Views.View\_ReportScreen.ReportWindow property)@\spxentry{del\_BtnFlyAgain()}\spxextra{src.Views.View\_ReportScreen.ReportWindow property}}

\begin{fulllineitems}
\phantomsection\label{\detokenize{index:src.Views.View_ReportScreen.ReportWindow.del_BtnFlyAgain}}\pysigline{\sphinxbfcode{\sphinxupquote{property }}\sphinxbfcode{\sphinxupquote{del\_BtnFlyAgain}}}
The fly again button for the view.
\begin{quote}\begin{description}
\item[{Returns}] \leavevmode
None

\end{description}\end{quote}

\end{fulllineitems}

\index{del\_BtnHome() (src.Views.View\_ReportScreen.ReportWindow property)@\spxentry{del\_BtnHome()}\spxextra{src.Views.View\_ReportScreen.ReportWindow property}}

\begin{fulllineitems}
\phantomsection\label{\detokenize{index:src.Views.View_ReportScreen.ReportWindow.del_BtnHome}}\pysigline{\sphinxbfcode{\sphinxupquote{property }}\sphinxbfcode{\sphinxupquote{del\_BtnHome}}}
The home for the view. Is used to return to home screen.
\begin{quote}\begin{description}
\item[{Returns}] \leavevmode
None

\end{description}\end{quote}

\end{fulllineitems}

\index{del\_BtnViewGraphNoVelocity() (src.Views.View\_ReportScreen.ReportWindow property)@\spxentry{del\_BtnViewGraphNoVelocity()}\spxextra{src.Views.View\_ReportScreen.ReportWindow property}}

\begin{fulllineitems}
\phantomsection\label{\detokenize{index:src.Views.View_ReportScreen.ReportWindow.del_BtnViewGraphNoVelocity}}\pysigline{\sphinxbfcode{\sphinxupquote{property }}\sphinxbfcode{\sphinxupquote{del\_BtnViewGraphNoVelocity}}}
The home for the view graph without velocity button.
\begin{quote}\begin{description}
\item[{Returns}] \leavevmode
None

\end{description}\end{quote}

\end{fulllineitems}

\index{del\_BtnViewGraphVelocity() (src.Views.View\_ReportScreen.ReportWindow property)@\spxentry{del\_BtnViewGraphVelocity()}\spxextra{src.Views.View\_ReportScreen.ReportWindow property}}

\begin{fulllineitems}
\phantomsection\label{\detokenize{index:src.Views.View_ReportScreen.ReportWindow.del_BtnViewGraphVelocity}}\pysigline{\sphinxbfcode{\sphinxupquote{property }}\sphinxbfcode{\sphinxupquote{del\_BtnViewGraphVelocity}}}
The home for the view graph with velocity button.
\begin{quote}\begin{description}
\item[{Returns}] \leavevmode
None

\end{description}\end{quote}

\end{fulllineitems}

\index{del\_LblFlightDate() (src.Views.View\_ReportScreen.ReportWindow property)@\spxentry{del\_LblFlightDate()}\spxextra{src.Views.View\_ReportScreen.ReportWindow property}}

\begin{fulllineitems}
\phantomsection\label{\detokenize{index:src.Views.View_ReportScreen.ReportWindow.del_LblFlightDate}}\pysigline{\sphinxbfcode{\sphinxupquote{property }}\sphinxbfcode{\sphinxupquote{del\_LblFlightDate}}}
Getter for the flight date label.
\begin{quote}\begin{description}
\item[{Returns}] \leavevmode
Reference to the flight date label.

\end{description}\end{quote}

\end{fulllineitems}

\index{del\_LblFlightInstructions() (src.Views.View\_ReportScreen.ReportWindow property)@\spxentry{del\_LblFlightInstructions()}\spxextra{src.Views.View\_ReportScreen.ReportWindow property}}

\begin{fulllineitems}
\phantomsection\label{\detokenize{index:src.Views.View_ReportScreen.ReportWindow.del_LblFlightInstructions}}\pysigline{\sphinxbfcode{\sphinxupquote{property }}\sphinxbfcode{\sphinxupquote{del\_LblFlightInstructions}}}
Getter for the flight instructions label.
\begin{quote}\begin{description}
\item[{Returns}] \leavevmode
Reference to the flight length label.

\end{description}\end{quote}

\end{fulllineitems}

\index{del\_LblFlightLength() (src.Views.View\_ReportScreen.ReportWindow property)@\spxentry{del\_LblFlightLength()}\spxextra{src.Views.View\_ReportScreen.ReportWindow property}}

\begin{fulllineitems}
\phantomsection\label{\detokenize{index:src.Views.View_ReportScreen.ReportWindow.del_LblFlightLength}}\pysigline{\sphinxbfcode{\sphinxupquote{property }}\sphinxbfcode{\sphinxupquote{del\_LblFlightLength}}}
Getter for the flight length label.
\begin{quote}\begin{description}
\item[{Returns}] \leavevmode
Reference to the flight length label.

\end{description}\end{quote}

\end{fulllineitems}

\index{del\_LblInstructor() (src.Views.View\_ReportScreen.ReportWindow property)@\spxentry{del\_LblInstructor()}\spxextra{src.Views.View\_ReportScreen.ReportWindow property}}

\begin{fulllineitems}
\phantomsection\label{\detokenize{index:src.Views.View_ReportScreen.ReportWindow.del_LblInstructor}}\pysigline{\sphinxbfcode{\sphinxupquote{property }}\sphinxbfcode{\sphinxupquote{del\_LblInstructor}}}
Getter for the instructor label.
\begin{quote}\begin{description}
\item[{Returns}] \leavevmode
Reference to the instructor label.

\end{description}\end{quote}

\end{fulllineitems}

\index{del\_LblPilot() (src.Views.View\_ReportScreen.ReportWindow property)@\spxentry{del\_LblPilot()}\spxextra{src.Views.View\_ReportScreen.ReportWindow property}}

\begin{fulllineitems}
\phantomsection\label{\detokenize{index:src.Views.View_ReportScreen.ReportWindow.del_LblPilot}}\pysigline{\sphinxbfcode{\sphinxupquote{property }}\sphinxbfcode{\sphinxupquote{del\_LblPilot}}}
Getter for the Pilot label so we can set who the pilot is for the flight in child class.
\begin{quote}\begin{description}
\item[{Returns}] \leavevmode
Reference to the pilot label.

\end{description}\end{quote}

\end{fulllineitems}

\index{handleEndSliderValueChange() (src.Views.View\_ReportScreen.ReportWindow method)@\spxentry{handleEndSliderValueChange()}\spxextra{src.Views.View\_ReportScreen.ReportWindow method}}

\begin{fulllineitems}
\phantomsection\label{\detokenize{index:src.Views.View_ReportScreen.ReportWindow.handleEndSliderValueChange}}\pysiglinewithargsret{\sphinxbfcode{\sphinxupquote{handleEndSliderValueChange}}}{\emph{value}}{{ $\rightarrow$ None}}
Listener that will updated the stop time label when the user moves the stop
time slider.
\begin{quote}\begin{description}
\item[{Parameters}] \leavevmode
\sphinxstyleliteralstrong{\sphinxupquote{value}} \textendash{} The value that the stop slider has been moved to horizontally.

\item[{Returns}] \leavevmode
None

\end{description}\end{quote}

\end{fulllineitems}

\index{handleStartSliderValueChange() (src.Views.View\_ReportScreen.ReportWindow method)@\spxentry{handleStartSliderValueChange()}\spxextra{src.Views.View\_ReportScreen.ReportWindow method}}

\begin{fulllineitems}
\phantomsection\label{\detokenize{index:src.Views.View_ReportScreen.ReportWindow.handleStartSliderValueChange}}\pysiglinewithargsret{\sphinxbfcode{\sphinxupquote{handleStartSliderValueChange}}}{\emph{value}}{{ $\rightarrow$ None}}
Listener that will updated the start time label when the user moves the start
time slider.
\begin{quote}\begin{description}
\item[{Parameters}] \leavevmode
\sphinxstyleliteralstrong{\sphinxupquote{value}} \textendash{} The value that the start slider has been moved to horizontally.

\item[{Returns}] \leavevmode
None

\end{description}\end{quote}

\end{fulllineitems}

\index{initView() (src.Views.View\_ReportScreen.ReportWindow method)@\spxentry{initView()}\spxextra{src.Views.View\_ReportScreen.ReportWindow method}}

\begin{fulllineitems}
\phantomsection\label{\detokenize{index:src.Views.View_ReportScreen.ReportWindow.initView}}\pysiglinewithargsret{\sphinxbfcode{\sphinxupquote{initView}}}{\emph{pilotName: str}, \emph{instructorName: str}, \emph{flightInstructions: str}, \emph{previousFlight: str}, \emph{usingPreviousFlight: bool}, \emph{flightData: dict}}{{ $\rightarrow$ None}}
Sets up the view and lays out all of the components.
\begin{quote}\begin{description}
\item[{Parameters}] \leavevmode\begin{itemize}
\item {} 
\sphinxstyleliteralstrong{\sphinxupquote{pilotName}} \textendash{} String containing pilot name

\item {} 
\sphinxstyleliteralstrong{\sphinxupquote{instructorName}} \textendash{} String containing instructor name

\item {} 
\sphinxstyleliteralstrong{\sphinxupquote{flightInstructions}} \textendash{} String containing flight instructions.

\item {} 
\sphinxstyleliteralstrong{\sphinxupquote{previousFlight}} \textendash{} String containing path to flight data. Should be .flight file if usingPreviousFlight is true, or blank if usingPreviousFlight is false.

\item {} 
\sphinxstyleliteralstrong{\sphinxupquote{usingPreviousFlight}} \textendash{} Boolean denoting if the report should be populated from the same file or a different one.

\item {} 
\sphinxstyleliteralstrong{\sphinxupquote{flightDict}} \textendash{} Dictionary containing flight data. Should be empty if usingPreviousFlight is true.

\end{itemize}

\item[{Returns}] \leavevmode
None

\end{description}\end{quote}

\end{fulllineitems}

\index{setButtonLayout() (src.Views.View\_ReportScreen.ReportWindow method)@\spxentry{setButtonLayout()}\spxextra{src.Views.View\_ReportScreen.ReportWindow method}}

\begin{fulllineitems}
\phantomsection\label{\detokenize{index:src.Views.View_ReportScreen.ReportWindow.setButtonLayout}}\pysiglinewithargsret{\sphinxbfcode{\sphinxupquote{setButtonLayout}}}{}{{ $\rightarrow$ PyQt5.QtWidgets.QHBoxLayout}}
Lays out the ‘Export Results’, ‘Fly Again’ and ‘Import Previous Flight’ buttons into a horizontal layout to be
put on screen.
\begin{quote}\begin{description}
\item[{Returns}] \leavevmode
The horizontal layout containing the 3 buttons

\end{description}\end{quote}

\end{fulllineitems}

\index{setSubTitle() (src.Views.View\_ReportScreen.ReportWindow method)@\spxentry{setSubTitle()}\spxextra{src.Views.View\_ReportScreen.ReportWindow method}}

\begin{fulllineitems}
\phantomsection\label{\detokenize{index:src.Views.View_ReportScreen.ReportWindow.setSubTitle}}\pysiglinewithargsret{\sphinxbfcode{\sphinxupquote{setSubTitle}}}{\emph{text}}{{ $\rightarrow$ PyQt5.QtWidgets.QLabel}}
Sets up a subtitle label for the window
\begin{quote}\begin{description}
\item[{Parameters}] \leavevmode
\sphinxstyleliteralstrong{\sphinxupquote{text}} \textendash{} String as name for label

\item[{Returns}] \leavevmode
Subtitle label

\end{description}\end{quote}

\end{fulllineitems}

\index{set\_BtnExport() (src.Views.View\_ReportScreen.ReportWindow property)@\spxentry{set\_BtnExport()}\spxextra{src.Views.View\_ReportScreen.ReportWindow property}}

\begin{fulllineitems}
\phantomsection\label{\detokenize{index:src.Views.View_ReportScreen.ReportWindow.set_BtnExport}}\pysigline{\sphinxbfcode{\sphinxupquote{property }}\sphinxbfcode{\sphinxupquote{set\_BtnExport}}}
The export button for the view.
\begin{quote}\begin{description}
\item[{Returns}] \leavevmode
None

\end{description}\end{quote}

\end{fulllineitems}

\index{set\_BtnFlyAgain() (src.Views.View\_ReportScreen.ReportWindow property)@\spxentry{set\_BtnFlyAgain()}\spxextra{src.Views.View\_ReportScreen.ReportWindow property}}

\begin{fulllineitems}
\phantomsection\label{\detokenize{index:src.Views.View_ReportScreen.ReportWindow.set_BtnFlyAgain}}\pysigline{\sphinxbfcode{\sphinxupquote{property }}\sphinxbfcode{\sphinxupquote{set\_BtnFlyAgain}}}
The fly again button for the view.
\begin{quote}\begin{description}
\item[{Returns}] \leavevmode
None

\end{description}\end{quote}

\end{fulllineitems}

\index{set\_BtnHome() (src.Views.View\_ReportScreen.ReportWindow property)@\spxentry{set\_BtnHome()}\spxextra{src.Views.View\_ReportScreen.ReportWindow property}}

\begin{fulllineitems}
\phantomsection\label{\detokenize{index:src.Views.View_ReportScreen.ReportWindow.set_BtnHome}}\pysigline{\sphinxbfcode{\sphinxupquote{property }}\sphinxbfcode{\sphinxupquote{set\_BtnHome}}}
The home for the view. Is used to return to home screen.
\begin{quote}\begin{description}
\item[{Returns}] \leavevmode
None

\end{description}\end{quote}

\end{fulllineitems}

\index{set\_BtnViewGraphNoVelocity() (src.Views.View\_ReportScreen.ReportWindow property)@\spxentry{set\_BtnViewGraphNoVelocity()}\spxextra{src.Views.View\_ReportScreen.ReportWindow property}}

\begin{fulllineitems}
\phantomsection\label{\detokenize{index:src.Views.View_ReportScreen.ReportWindow.set_BtnViewGraphNoVelocity}}\pysigline{\sphinxbfcode{\sphinxupquote{property }}\sphinxbfcode{\sphinxupquote{set\_BtnViewGraphNoVelocity}}}
The home for the view graph without velocity button.
\begin{quote}\begin{description}
\item[{Returns}] \leavevmode
None

\end{description}\end{quote}

\end{fulllineitems}

\index{set\_BtnViewGraphVelocity() (src.Views.View\_ReportScreen.ReportWindow property)@\spxentry{set\_BtnViewGraphVelocity()}\spxextra{src.Views.View\_ReportScreen.ReportWindow property}}

\begin{fulllineitems}
\phantomsection\label{\detokenize{index:src.Views.View_ReportScreen.ReportWindow.set_BtnViewGraphVelocity}}\pysigline{\sphinxbfcode{\sphinxupquote{property }}\sphinxbfcode{\sphinxupquote{set\_BtnViewGraphVelocity}}}
The home for the view graph with velocity button.
\begin{quote}\begin{description}
\item[{Returns}] \leavevmode
None

\end{description}\end{quote}

\end{fulllineitems}

\index{set\_LblFlightDate() (src.Views.View\_ReportScreen.ReportWindow property)@\spxentry{set\_LblFlightDate()}\spxextra{src.Views.View\_ReportScreen.ReportWindow property}}

\begin{fulllineitems}
\phantomsection\label{\detokenize{index:src.Views.View_ReportScreen.ReportWindow.set_LblFlightDate}}\pysigline{\sphinxbfcode{\sphinxupquote{property }}\sphinxbfcode{\sphinxupquote{set\_LblFlightDate}}}
Getter for the flight date label.
\begin{quote}\begin{description}
\item[{Returns}] \leavevmode
Reference to the flight date label.

\end{description}\end{quote}

\end{fulllineitems}

\index{set\_LblFlightInstructions() (src.Views.View\_ReportScreen.ReportWindow property)@\spxentry{set\_LblFlightInstructions()}\spxextra{src.Views.View\_ReportScreen.ReportWindow property}}

\begin{fulllineitems}
\phantomsection\label{\detokenize{index:src.Views.View_ReportScreen.ReportWindow.set_LblFlightInstructions}}\pysigline{\sphinxbfcode{\sphinxupquote{property }}\sphinxbfcode{\sphinxupquote{set\_LblFlightInstructions}}}
Getter for the flight instructions label.
\begin{quote}\begin{description}
\item[{Returns}] \leavevmode
Reference to the flight length label.

\end{description}\end{quote}

\end{fulllineitems}

\index{set\_LblFlightLength() (src.Views.View\_ReportScreen.ReportWindow property)@\spxentry{set\_LblFlightLength()}\spxextra{src.Views.View\_ReportScreen.ReportWindow property}}

\begin{fulllineitems}
\phantomsection\label{\detokenize{index:src.Views.View_ReportScreen.ReportWindow.set_LblFlightLength}}\pysigline{\sphinxbfcode{\sphinxupquote{property }}\sphinxbfcode{\sphinxupquote{set\_LblFlightLength}}}
Getter for the flight length label.
\begin{quote}\begin{description}
\item[{Returns}] \leavevmode
Reference to the flight length label.

\end{description}\end{quote}

\end{fulllineitems}

\index{set\_LblInstructor() (src.Views.View\_ReportScreen.ReportWindow property)@\spxentry{set\_LblInstructor()}\spxextra{src.Views.View\_ReportScreen.ReportWindow property}}

\begin{fulllineitems}
\phantomsection\label{\detokenize{index:src.Views.View_ReportScreen.ReportWindow.set_LblInstructor}}\pysigline{\sphinxbfcode{\sphinxupquote{property }}\sphinxbfcode{\sphinxupquote{set\_LblInstructor}}}
Getter for the instructor label.
\begin{quote}\begin{description}
\item[{Returns}] \leavevmode
Reference to the instructor label.

\end{description}\end{quote}

\end{fulllineitems}

\index{set\_LblPilot() (src.Views.View\_ReportScreen.ReportWindow property)@\spxentry{set\_LblPilot()}\spxextra{src.Views.View\_ReportScreen.ReportWindow property}}

\begin{fulllineitems}
\phantomsection\label{\detokenize{index:src.Views.View_ReportScreen.ReportWindow.set_LblPilot}}\pysigline{\sphinxbfcode{\sphinxupquote{property }}\sphinxbfcode{\sphinxupquote{set\_LblPilot}}}
Getter for the Pilot label so we can set who the pilot is for the flight in child class.
\begin{quote}\begin{description}
\item[{Returns}] \leavevmode
Reference to the pilot label.

\end{description}\end{quote}

\end{fulllineitems}

\index{setupFlightInfo() (src.Views.View\_ReportScreen.ReportWindow method)@\spxentry{setupFlightInfo()}\spxextra{src.Views.View\_ReportScreen.ReportWindow method}}

\begin{fulllineitems}
\phantomsection\label{\detokenize{index:src.Views.View_ReportScreen.ReportWindow.setupFlightInfo}}\pysiglinewithargsret{\sphinxbfcode{\sphinxupquote{setupFlightInfo}}}{}{{ $\rightarrow$ PyQt5.QtWidgets.QGridLayout}}
Sets up the flight info (pilot, instructor, date, length, and smoothness score) in a grid.
\begin{quote}\begin{description}
\item[{Returns}] \leavevmode
Grid layout of the flight information

\end{description}\end{quote}

\end{fulllineitems}

\index{setupGraph() (src.Views.View\_ReportScreen.ReportWindow method)@\spxentry{setupGraph()}\spxextra{src.Views.View\_ReportScreen.ReportWindow method}}

\begin{fulllineitems}
\phantomsection\label{\detokenize{index:src.Views.View_ReportScreen.ReportWindow.setupGraph}}\pysiglinewithargsret{\sphinxbfcode{\sphinxupquote{setupGraph}}}{\emph{flightData: dict}, \emph{displayVelocity: bool}}{{ $\rightarrow$ None}}
Sets up the 3d plot for viewing upon click of button. displayVelocity is a boolean denoting if the graph should
display colored segments for velocity.
\begin{quote}\begin{description}
\item[{Parameters}] \leavevmode\begin{itemize}
\item {} 
\sphinxstyleliteralstrong{\sphinxupquote{flightData}} \textendash{} Dictionary of flight data

\item {} 
\sphinxstyleliteralstrong{\sphinxupquote{displayVelocity}} \textendash{} Bool denoting if velocity should be plotted or not.

\end{itemize}

\item[{Returns}] \leavevmode
None

\end{description}\end{quote}

\end{fulllineitems}

\index{setupSlider() (src.Views.View\_ReportScreen.ReportWindow method)@\spxentry{setupSlider()}\spxextra{src.Views.View\_ReportScreen.ReportWindow method}}

\begin{fulllineitems}
\phantomsection\label{\detokenize{index:src.Views.View_ReportScreen.ReportWindow.setupSlider}}\pysiglinewithargsret{\sphinxbfcode{\sphinxupquote{setupSlider}}}{}{{ $\rightarrow$ PyQt5.QtWidgets.QVBoxLayout}}
Setups the slider that will be used to adjust the times of the flight path that will be displayed
on the graph. This lets us control the beginning and ending bounds of the time of the flight
that we want to view. We have to use 2 sliders. One for the start time and one for the end.
We originally wanted to make this such that both sliders were on top of each other to make
it more intuitive, but we ran into a bug that prevented us from doing that.
\begin{quote}\begin{description}
\item[{Returns}] \leavevmode
A horizontal layout containing both sliders and the labels displaying the time that they represent.

\end{description}\end{quote}

\end{fulllineitems}

\index{setupTitle() (src.Views.View\_ReportScreen.ReportWindow method)@\spxentry{setupTitle()}\spxextra{src.Views.View\_ReportScreen.ReportWindow method}}

\begin{fulllineitems}
\phantomsection\label{\detokenize{index:src.Views.View_ReportScreen.ReportWindow.setupTitle}}\pysiglinewithargsret{\sphinxbfcode{\sphinxupquote{setupTitle}}}{}{{ $\rightarrow$ PyQt5.QtWidgets.QVBoxLayout}}
Sets up the title with the application title on top and the name of the screen just below it.
\begin{quote}\begin{description}
\item[{Returns}] \leavevmode
Layout with the application title and screen title labels

\end{description}\end{quote}

\end{fulllineitems}

\index{showWindow() (src.Views.View\_ReportScreen.ReportWindow method)@\spxentry{showWindow()}\spxextra{src.Views.View\_ReportScreen.ReportWindow method}}

\begin{fulllineitems}
\phantomsection\label{\detokenize{index:src.Views.View_ReportScreen.ReportWindow.showWindow}}\pysiglinewithargsret{\sphinxbfcode{\sphinxupquote{showWindow}}}{}{{ $\rightarrow$ None}}
Takes all of the elements from the view and displays the window.
\begin{quote}\begin{description}
\item[{Returns}] \leavevmode
None

\end{description}\end{quote}

\end{fulllineitems}

\index{signalExportResults() (src.Views.View\_ReportScreen.ReportWindow method)@\spxentry{signalExportResults()}\spxextra{src.Views.View\_ReportScreen.ReportWindow method}}

\begin{fulllineitems}
\phantomsection\label{\detokenize{index:src.Views.View_ReportScreen.ReportWindow.signalExportResults}}\pysiglinewithargsret{\sphinxbfcode{\sphinxupquote{signalExportResults}}}{}{{ $\rightarrow$ None}}
Sends a signal to the main controller that the Export Results button was pushed.
\begin{quote}\begin{description}
\item[{Returns}] \leavevmode
none

\end{description}\end{quote}

\end{fulllineitems}

\index{signalReturnHome() (src.Views.View\_ReportScreen.ReportWindow method)@\spxentry{signalReturnHome()}\spxextra{src.Views.View\_ReportScreen.ReportWindow method}}

\begin{fulllineitems}
\phantomsection\label{\detokenize{index:src.Views.View_ReportScreen.ReportWindow.signalReturnHome}}\pysiglinewithargsret{\sphinxbfcode{\sphinxupquote{signalReturnHome}}}{}{{ $\rightarrow$ None}}
Sends a signal to the main controller that the Return Home button was pushed.
\begin{quote}\begin{description}
\item[{Returns}] \leavevmode
none

\end{description}\end{quote}

\end{fulllineitems}

\index{signalStartTracking() (src.Views.View\_ReportScreen.ReportWindow method)@\spxentry{signalStartTracking()}\spxextra{src.Views.View\_ReportScreen.ReportWindow method}}

\begin{fulllineitems}
\phantomsection\label{\detokenize{index:src.Views.View_ReportScreen.ReportWindow.signalStartTracking}}\pysiglinewithargsret{\sphinxbfcode{\sphinxupquote{signalStartTracking}}}{}{{ $\rightarrow$ None}}
Sends a signal to the main controller that the Fly Again button was pushed.
\begin{quote}\begin{description}
\item[{Returns}] \leavevmode
none

\end{description}\end{quote}

\end{fulllineitems}


\end{fulllineitems}



\chapter{Startup Screen}
\label{\detokenize{index:module-src.Views.View_StartupScreen}}\label{\detokenize{index:startup-screen}}\index{src.Views.View\_StartupScreen (module)@\spxentry{src.Views.View\_StartupScreen}\spxextra{module}}\index{StartupWindow (class in src.Views.View\_StartupScreen)@\spxentry{StartupWindow}\spxextra{class in src.Views.View\_StartupScreen}}

\begin{fulllineitems}
\phantomsection\label{\detokenize{index:src.Views.View_StartupScreen.StartupWindow}}\pysiglinewithargsret{\sphinxbfcode{\sphinxupquote{class }}\sphinxcode{\sphinxupquote{src.Views.View\_StartupScreen.}}\sphinxbfcode{\sphinxupquote{StartupWindow}}}{\emph{flightModeEnabled: bool}}{}
The view for the home Startup page that is shown when the user opens the application.
\begin{quote}\begin{description}
\item[{Variables}] \leavevmode\begin{itemize}
\item {} 
\sphinxstyleliteralstrong{\sphinxupquote{\_\_btnVerifySetup}} \textendash{} The class property for the ‘Verify Setup’ button.

\item {} 
\sphinxstyleliteralstrong{\sphinxupquote{\_\_btnStart}} \textendash{} The class property for the ‘Start Tracking’ button.

\item {} 
\sphinxstyleliteralstrong{\sphinxupquote{\_\_btnImport}} \textendash{} The class property for the ‘Import Previous Flight’ button.

\end{itemize}

\end{description}\end{quote}
\index{BtnImport() (src.Views.View\_StartupScreen.StartupWindow property)@\spxentry{BtnImport()}\spxextra{src.Views.View\_StartupScreen.StartupWindow property}}

\begin{fulllineitems}
\phantomsection\label{\detokenize{index:src.Views.View_StartupScreen.StartupWindow.BtnImport}}\pysigline{\sphinxbfcode{\sphinxupquote{property }}\sphinxbfcode{\sphinxupquote{BtnImport}}}
Getter for the Import Previous Flight button. Is used to import past flight files. Use to attach functionality.
\begin{quote}\begin{description}
\item[{Returns}] \leavevmode
None

\end{description}\end{quote}

\end{fulllineitems}

\index{BtnStart() (src.Views.View\_StartupScreen.StartupWindow property)@\spxentry{BtnStart()}\spxextra{src.Views.View\_StartupScreen.StartupWindow property}}

\begin{fulllineitems}
\phantomsection\label{\detokenize{index:src.Views.View_StartupScreen.StartupWindow.BtnStart}}\pysigline{\sphinxbfcode{\sphinxupquote{property }}\sphinxbfcode{\sphinxupquote{BtnStart}}}
Getter for the startTracking button. Use to attach functionality.
\begin{quote}\begin{description}
\item[{Returns}] \leavevmode
None

\end{description}\end{quote}

\end{fulllineitems}

\index{BtnVerifySetup() (src.Views.View\_StartupScreen.StartupWindow property)@\spxentry{BtnVerifySetup()}\spxextra{src.Views.View\_StartupScreen.StartupWindow property}}

\begin{fulllineitems}
\phantomsection\label{\detokenize{index:src.Views.View_StartupScreen.StartupWindow.BtnVerifySetup}}\pysigline{\sphinxbfcode{\sphinxupquote{property }}\sphinxbfcode{\sphinxupquote{BtnVerifySetup}}}
Getter for the verifySetup button. Use to attach functionality.
\begin{quote}\begin{description}
\item[{Returns}] \leavevmode
The reference to the verifySetup button

\end{description}\end{quote}

\end{fulllineitems}

\index{del\_BtnImport() (src.Views.View\_StartupScreen.StartupWindow property)@\spxentry{del\_BtnImport()}\spxextra{src.Views.View\_StartupScreen.StartupWindow property}}

\begin{fulllineitems}
\phantomsection\label{\detokenize{index:src.Views.View_StartupScreen.StartupWindow.del_BtnImport}}\pysigline{\sphinxbfcode{\sphinxupquote{property }}\sphinxbfcode{\sphinxupquote{del\_BtnImport}}}
Getter for the Import Previous Flight button. Is used to import past flight files. Use to attach functionality.
\begin{quote}\begin{description}
\item[{Returns}] \leavevmode
None

\end{description}\end{quote}

\end{fulllineitems}

\index{del\_BtnStart() (src.Views.View\_StartupScreen.StartupWindow property)@\spxentry{del\_BtnStart()}\spxextra{src.Views.View\_StartupScreen.StartupWindow property}}

\begin{fulllineitems}
\phantomsection\label{\detokenize{index:src.Views.View_StartupScreen.StartupWindow.del_BtnStart}}\pysigline{\sphinxbfcode{\sphinxupquote{property }}\sphinxbfcode{\sphinxupquote{del\_BtnStart}}}
Getter for the startTracking button. Use to attach functionality.
\begin{quote}\begin{description}
\item[{Returns}] \leavevmode
None

\end{description}\end{quote}

\end{fulllineitems}

\index{del\_BtnVerifySetup() (src.Views.View\_StartupScreen.StartupWindow property)@\spxentry{del\_BtnVerifySetup()}\spxextra{src.Views.View\_StartupScreen.StartupWindow property}}

\begin{fulllineitems}
\phantomsection\label{\detokenize{index:src.Views.View_StartupScreen.StartupWindow.del_BtnVerifySetup}}\pysigline{\sphinxbfcode{\sphinxupquote{property }}\sphinxbfcode{\sphinxupquote{del\_BtnVerifySetup}}}
Getter for the verifySetup button. Use to attach functionality.
\begin{quote}\begin{description}
\item[{Returns}] \leavevmode
The reference to the verifySetup button

\end{description}\end{quote}

\end{fulllineitems}

\index{initView() (src.Views.View\_StartupScreen.StartupWindow method)@\spxentry{initView()}\spxextra{src.Views.View\_StartupScreen.StartupWindow method}}

\begin{fulllineitems}
\phantomsection\label{\detokenize{index:src.Views.View_StartupScreen.StartupWindow.initView}}\pysiglinewithargsret{\sphinxbfcode{\sphinxupquote{initView}}}{}{{ $\rightarrow$ None}}
Sets up the view and lays out all of the components.
\begin{quote}\begin{description}
\item[{Returns}] \leavevmode
None

\end{description}\end{quote}

\end{fulllineitems}

\index{openFileNameDialog() (src.Views.View\_StartupScreen.StartupWindow method)@\spxentry{openFileNameDialog()}\spxextra{src.Views.View\_StartupScreen.StartupWindow method}}

\begin{fulllineitems}
\phantomsection\label{\detokenize{index:src.Views.View_StartupScreen.StartupWindow.openFileNameDialog}}\pysiglinewithargsret{\sphinxbfcode{\sphinxupquote{openFileNameDialog}}}{}{{ $\rightarrow$ None}}
Allows user to select a .flight file from a file dialog window.
\begin{quote}\begin{description}
\item[{Returns}] \leavevmode
Path to selected file as a string.

\end{description}\end{quote}

\end{fulllineitems}

\index{setButtonLayout() (src.Views.View\_StartupScreen.StartupWindow method)@\spxentry{setButtonLayout()}\spxextra{src.Views.View\_StartupScreen.StartupWindow method}}

\begin{fulllineitems}
\phantomsection\label{\detokenize{index:src.Views.View_StartupScreen.StartupWindow.setButtonLayout}}\pysiglinewithargsret{\sphinxbfcode{\sphinxupquote{setButtonLayout}}}{}{{ $\rightarrow$ PyQt5.QtWidgets.QHBoxLayout}}
Lays out the ‘Test Config’, ‘Start’ and ‘Import’ buttons into a horizontal layout to be
put on screen.
\begin{quote}\begin{description}
\item[{Returns}] \leavevmode
The horizontal layout containing the 3 buttons

\end{description}\end{quote}

\end{fulllineitems}

\index{setTeamMembers() (src.Views.View\_StartupScreen.StartupWindow method)@\spxentry{setTeamMembers()}\spxextra{src.Views.View\_StartupScreen.StartupWindow method}}

\begin{fulllineitems}
\phantomsection\label{\detokenize{index:src.Views.View_StartupScreen.StartupWindow.setTeamMembers}}\pysiglinewithargsret{\sphinxbfcode{\sphinxupquote{setTeamMembers}}}{}{{ $\rightarrow$ PyQt5.QtWidgets.QVBoxLayout}}
Sets up the team members label for the window
\begin{quote}\begin{description}
\item[{Returns}] \leavevmode
Team members label of the application

\end{description}\end{quote}

\end{fulllineitems}

\index{setTitle() (src.Views.View\_StartupScreen.StartupWindow method)@\spxentry{setTitle()}\spxextra{src.Views.View\_StartupScreen.StartupWindow method}}

\begin{fulllineitems}
\phantomsection\label{\detokenize{index:src.Views.View_StartupScreen.StartupWindow.setTitle}}\pysiglinewithargsret{\sphinxbfcode{\sphinxupquote{setTitle}}}{}{{ $\rightarrow$ PyQt5.QtWidgets.QVBoxLayout}}
Sets up the title with the application title on top and the name of the screen just below it.
\begin{quote}\begin{description}
\item[{Returns}] \leavevmode
Layout with the application title and screen title labels

\end{description}\end{quote}

\end{fulllineitems}

\index{set\_BtnImport() (src.Views.View\_StartupScreen.StartupWindow property)@\spxentry{set\_BtnImport()}\spxextra{src.Views.View\_StartupScreen.StartupWindow property}}

\begin{fulllineitems}
\phantomsection\label{\detokenize{index:src.Views.View_StartupScreen.StartupWindow.set_BtnImport}}\pysigline{\sphinxbfcode{\sphinxupquote{property }}\sphinxbfcode{\sphinxupquote{set\_BtnImport}}}
Getter for the Import Previous Flight button. Is used to import past flight files. Use to attach functionality.
\begin{quote}\begin{description}
\item[{Returns}] \leavevmode
None

\end{description}\end{quote}

\end{fulllineitems}

\index{set\_BtnStart() (src.Views.View\_StartupScreen.StartupWindow property)@\spxentry{set\_BtnStart()}\spxextra{src.Views.View\_StartupScreen.StartupWindow property}}

\begin{fulllineitems}
\phantomsection\label{\detokenize{index:src.Views.View_StartupScreen.StartupWindow.set_BtnStart}}\pysigline{\sphinxbfcode{\sphinxupquote{property }}\sphinxbfcode{\sphinxupquote{set\_BtnStart}}}
Getter for the startTracking button. Use to attach functionality.
\begin{quote}\begin{description}
\item[{Returns}] \leavevmode
None

\end{description}\end{quote}

\end{fulllineitems}

\index{set\_BtnVerifySetup() (src.Views.View\_StartupScreen.StartupWindow property)@\spxentry{set\_BtnVerifySetup()}\spxextra{src.Views.View\_StartupScreen.StartupWindow property}}

\begin{fulllineitems}
\phantomsection\label{\detokenize{index:src.Views.View_StartupScreen.StartupWindow.set_BtnVerifySetup}}\pysigline{\sphinxbfcode{\sphinxupquote{property }}\sphinxbfcode{\sphinxupquote{set\_BtnVerifySetup}}}
Getter for the verifySetup button. Use to attach functionality.
\begin{quote}\begin{description}
\item[{Returns}] \leavevmode
The reference to the verifySetup button

\end{description}\end{quote}

\end{fulllineitems}

\index{setupAMLogo() (src.Views.View\_StartupScreen.StartupWindow method)@\spxentry{setupAMLogo()}\spxextra{src.Views.View\_StartupScreen.StartupWindow method}}

\begin{fulllineitems}
\phantomsection\label{\detokenize{index:src.Views.View_StartupScreen.StartupWindow.setupAMLogo}}\pysiglinewithargsret{\sphinxbfcode{\sphinxupquote{setupAMLogo}}}{}{{ $\rightarrow$ None}}
Used for configuring the display for the A\&M logo on the startup screen.
\begin{quote}\begin{description}
\item[{Returns}] \leavevmode
None

\end{description}\end{quote}

\end{fulllineitems}

\index{setupPicture() (src.Views.View\_StartupScreen.StartupWindow method)@\spxentry{setupPicture()}\spxextra{src.Views.View\_StartupScreen.StartupWindow method}}

\begin{fulllineitems}
\phantomsection\label{\detokenize{index:src.Views.View_StartupScreen.StartupWindow.setupPicture}}\pysiglinewithargsret{\sphinxbfcode{\sphinxupquote{setupPicture}}}{}{{ $\rightarrow$ None}}
Used for configuring the display for the logo on the startup screen.
\begin{quote}\begin{description}
\item[{Returns}] \leavevmode
None

\end{description}\end{quote}

\end{fulllineitems}

\index{signalImportFlight() (src.Views.View\_StartupScreen.StartupWindow method)@\spxentry{signalImportFlight()}\spxextra{src.Views.View\_StartupScreen.StartupWindow method}}

\begin{fulllineitems}
\phantomsection\label{\detokenize{index:src.Views.View_StartupScreen.StartupWindow.signalImportFlight}}\pysiglinewithargsret{\sphinxbfcode{\sphinxupquote{signalImportFlight}}}{}{{ $\rightarrow$ None}}
Calls function to allow user to select a file for import.
Sends a signal to the main controller that the Import Previous Flight button was pushed.
\begin{quote}\begin{description}
\item[{Returns}] \leavevmode
None.

\end{description}\end{quote}

\end{fulllineitems}

\index{signalStartTracking() (src.Views.View\_StartupScreen.StartupWindow method)@\spxentry{signalStartTracking()}\spxextra{src.Views.View\_StartupScreen.StartupWindow method}}

\begin{fulllineitems}
\phantomsection\label{\detokenize{index:src.Views.View_StartupScreen.StartupWindow.signalStartTracking}}\pysiglinewithargsret{\sphinxbfcode{\sphinxupquote{signalStartTracking}}}{}{{ $\rightarrow$ None}}
Sends a signal to the main controller that the Start Tracking button was pushed.
\begin{quote}\begin{description}
\item[{Returns}] \leavevmode
none

\end{description}\end{quote}

\end{fulllineitems}

\index{signalVerifySetup() (src.Views.View\_StartupScreen.StartupWindow method)@\spxentry{signalVerifySetup()}\spxextra{src.Views.View\_StartupScreen.StartupWindow method}}

\begin{fulllineitems}
\phantomsection\label{\detokenize{index:src.Views.View_StartupScreen.StartupWindow.signalVerifySetup}}\pysiglinewithargsret{\sphinxbfcode{\sphinxupquote{signalVerifySetup}}}{}{{ $\rightarrow$ None}}
Sends a signal to the main controller that the Verify Setup button was pushed.
\begin{quote}\begin{description}
\item[{Returns}] \leavevmode
none

\end{description}\end{quote}

\end{fulllineitems}


\end{fulllineitems}



\chapter{Tracking Screen}
\label{\detokenize{index:module-src.Views.View_TrackingScreen}}\label{\detokenize{index:tracking-screen}}\index{src.Views.View\_TrackingScreen (module)@\spxentry{src.Views.View\_TrackingScreen}\spxextra{module}}\index{TrackingWindow (class in src.Views.View\_TrackingScreen)@\spxentry{TrackingWindow}\spxextra{class in src.Views.View\_TrackingScreen}}

\begin{fulllineitems}
\phantomsection\label{\detokenize{index:src.Views.View_TrackingScreen.TrackingWindow}}\pysiglinewithargsret{\sphinxbfcode{\sphinxupquote{class }}\sphinxcode{\sphinxupquote{src.Views.View\_TrackingScreen.}}\sphinxbfcode{\sphinxupquote{TrackingWindow}}}{\emph{phoneControl: Controllers.PhoneController.PhoneControl}}{}
The view for the tracking view page that is shown when the user presses the “Start Tracking” button on the home page.
Allows the user to enter in flight information and begin tracking the drone.
\begin{quote}\begin{description}
\item[{Variables}] \leavevmode\begin{itemize}
\item {} 
\sphinxstyleliteralstrong{\sphinxupquote{\_\_btnConfirm}} \textendash{} The class property for the ‘Confirm’ button.

\item {} 
\sphinxstyleliteralstrong{\sphinxupquote{\_\_btnClear}} \textendash{} The class property for the ‘Clear’ button.

\item {} 
\sphinxstyleliteralstrong{\sphinxupquote{\_\_btnStart}} \textendash{} The class property for the ‘Start Tracking’ button.

\item {} 
\sphinxstyleliteralstrong{\sphinxupquote{\_\_btnStop}} \textendash{} The class property for the ‘Stop Tracking’ button.

\end{itemize}

\end{description}\end{quote}
\index{BtnClear() (src.Views.View\_TrackingScreen.TrackingWindow property)@\spxentry{BtnClear()}\spxextra{src.Views.View\_TrackingScreen.TrackingWindow property}}

\begin{fulllineitems}
\phantomsection\label{\detokenize{index:src.Views.View_TrackingScreen.TrackingWindow.BtnClear}}\pysigline{\sphinxbfcode{\sphinxupquote{property }}\sphinxbfcode{\sphinxupquote{BtnClear}}}
Getter for the Clear button so we can attach functionality to it later.
\begin{quote}\begin{description}
\item[{Returns}] \leavevmode
Reference to the clear button

\end{description}\end{quote}

\end{fulllineitems}

\index{BtnConfirm() (src.Views.View\_TrackingScreen.TrackingWindow property)@\spxentry{BtnConfirm()}\spxextra{src.Views.View\_TrackingScreen.TrackingWindow property}}

\begin{fulllineitems}
\phantomsection\label{\detokenize{index:src.Views.View_TrackingScreen.TrackingWindow.BtnConfirm}}\pysigline{\sphinxbfcode{\sphinxupquote{property }}\sphinxbfcode{\sphinxupquote{BtnConfirm}}}
Getter for the Confirm button so we can attach functionality to it later.
\begin{quote}\begin{description}
\item[{Returns}] \leavevmode
Reference to the confirm button

\end{description}\end{quote}

\end{fulllineitems}

\index{BtnStart() (src.Views.View\_TrackingScreen.TrackingWindow property)@\spxentry{BtnStart()}\spxextra{src.Views.View\_TrackingScreen.TrackingWindow property}}

\begin{fulllineitems}
\phantomsection\label{\detokenize{index:src.Views.View_TrackingScreen.TrackingWindow.BtnStart}}\pysigline{\sphinxbfcode{\sphinxupquote{property }}\sphinxbfcode{\sphinxupquote{BtnStart}}}
Getter for the Start button
\begin{quote}\begin{description}
\item[{Returns}] \leavevmode
Reference to the start button

\end{description}\end{quote}

\end{fulllineitems}

\index{BtnStop() (src.Views.View\_TrackingScreen.TrackingWindow property)@\spxentry{BtnStop()}\spxextra{src.Views.View\_TrackingScreen.TrackingWindow property}}

\begin{fulllineitems}
\phantomsection\label{\detokenize{index:src.Views.View_TrackingScreen.TrackingWindow.BtnStop}}\pysigline{\sphinxbfcode{\sphinxupquote{property }}\sphinxbfcode{\sphinxupquote{BtnStop}}}
Getter for the Stop button
\begin{quote}\begin{description}
\item[{Returns}] \leavevmode
Reference to the stop button

\end{description}\end{quote}

\end{fulllineitems}

\index{LblInstructor() (src.Views.View\_TrackingScreen.TrackingWindow property)@\spxentry{LblInstructor()}\spxextra{src.Views.View\_TrackingScreen.TrackingWindow property}}

\begin{fulllineitems}
\phantomsection\label{\detokenize{index:src.Views.View_TrackingScreen.TrackingWindow.LblInstructor}}\pysigline{\sphinxbfcode{\sphinxupquote{property }}\sphinxbfcode{\sphinxupquote{LblInstructor}}}
Getter for the Instructor label so we can attach functionality to it later.
\begin{quote}\begin{description}
\item[{Returns}] \leavevmode
The instructor label

\end{description}\end{quote}

\end{fulllineitems}

\index{LblPilot() (src.Views.View\_TrackingScreen.TrackingWindow property)@\spxentry{LblPilot()}\spxextra{src.Views.View\_TrackingScreen.TrackingWindow property}}

\begin{fulllineitems}
\phantomsection\label{\detokenize{index:src.Views.View_TrackingScreen.TrackingWindow.LblPilot}}\pysigline{\sphinxbfcode{\sphinxupquote{property }}\sphinxbfcode{\sphinxupquote{LblPilot}}}
Getter for the Pilot label so we can attach functionality to it
\begin{quote}\begin{description}
\item[{Returns}] \leavevmode
The pilot label

\end{description}\end{quote}

\end{fulllineitems}

\index{LblTimer() (src.Views.View\_TrackingScreen.TrackingWindow property)@\spxentry{LblTimer()}\spxextra{src.Views.View\_TrackingScreen.TrackingWindow property}}

\begin{fulllineitems}
\phantomsection\label{\detokenize{index:src.Views.View_TrackingScreen.TrackingWindow.LblTimer}}\pysigline{\sphinxbfcode{\sphinxupquote{property }}\sphinxbfcode{\sphinxupquote{LblTimer}}}
Getter property for the timer label. We need to attach a QTimer to it so it can count the time the
application has been tracking the drone.
\begin{quote}\begin{description}
\item[{Returns}] \leavevmode
The timer label

\end{description}\end{quote}

\end{fulllineitems}

\index{TBInstructor() (src.Views.View\_TrackingScreen.TrackingWindow property)@\spxentry{TBInstructor()}\spxextra{src.Views.View\_TrackingScreen.TrackingWindow property}}

\begin{fulllineitems}
\phantomsection\label{\detokenize{index:src.Views.View_TrackingScreen.TrackingWindow.TBInstructor}}\pysigline{\sphinxbfcode{\sphinxupquote{property }}\sphinxbfcode{\sphinxupquote{TBInstructor}}}
Getter for the Instructor textbox so we can attach functionality to it later.
\begin{quote}\begin{description}
\item[{Returns}] \leavevmode
The instructor textbox

\end{description}\end{quote}

\end{fulllineitems}

\index{TBPilot() (src.Views.View\_TrackingScreen.TrackingWindow property)@\spxentry{TBPilot()}\spxextra{src.Views.View\_TrackingScreen.TrackingWindow property}}

\begin{fulllineitems}
\phantomsection\label{\detokenize{index:src.Views.View_TrackingScreen.TrackingWindow.TBPilot}}\pysigline{\sphinxbfcode{\sphinxupquote{property }}\sphinxbfcode{\sphinxupquote{TBPilot}}}
Getter for the Pilot Textbox so we can attach functionality to it
\begin{quote}\begin{description}
\item[{Returns}] \leavevmode
The pilot textbox

\end{description}\end{quote}

\end{fulllineitems}

\index{TEInstructions() (src.Views.View\_TrackingScreen.TrackingWindow property)@\spxentry{TEInstructions()}\spxextra{src.Views.View\_TrackingScreen.TrackingWindow property}}

\begin{fulllineitems}
\phantomsection\label{\detokenize{index:src.Views.View_TrackingScreen.TrackingWindow.TEInstructions}}\pysigline{\sphinxbfcode{\sphinxupquote{property }}\sphinxbfcode{\sphinxupquote{TEInstructions}}}
Getter for the instructions text edit box
\begin{quote}\begin{description}
\item[{Returns}] \leavevmode
Reference to the instructions text edit box

\end{description}\end{quote}

\end{fulllineitems}

\index{clearValues() (src.Views.View\_TrackingScreen.TrackingWindow method)@\spxentry{clearValues()}\spxextra{src.Views.View\_TrackingScreen.TrackingWindow method}}

\begin{fulllineitems}
\phantomsection\label{\detokenize{index:src.Views.View_TrackingScreen.TrackingWindow.clearValues}}\pysiglinewithargsret{\sphinxbfcode{\sphinxupquote{clearValues}}}{}{{ $\rightarrow$ None}}
Clears the values in the text boxes.
\begin{quote}\begin{description}
\item[{Returns}] \leavevmode
None

\end{description}\end{quote}

\end{fulllineitems}

\index{confirmValues() (src.Views.View\_TrackingScreen.TrackingWindow method)@\spxentry{confirmValues()}\spxextra{src.Views.View\_TrackingScreen.TrackingWindow method}}

\begin{fulllineitems}
\phantomsection\label{\detokenize{index:src.Views.View_TrackingScreen.TrackingWindow.confirmValues}}\pysiglinewithargsret{\sphinxbfcode{\sphinxupquote{confirmValues}}}{}{{ $\rightarrow$ None}}
Confirms the values in the textboxes by displaying a pop up message of the values.
\begin{quote}\begin{description}
\item[{Returns}] \leavevmode
None

\end{description}\end{quote}

\end{fulllineitems}

\index{del\_BtnClear() (src.Views.View\_TrackingScreen.TrackingWindow property)@\spxentry{del\_BtnClear()}\spxextra{src.Views.View\_TrackingScreen.TrackingWindow property}}

\begin{fulllineitems}
\phantomsection\label{\detokenize{index:src.Views.View_TrackingScreen.TrackingWindow.del_BtnClear}}\pysigline{\sphinxbfcode{\sphinxupquote{property }}\sphinxbfcode{\sphinxupquote{del\_BtnClear}}}
Getter for the Clear button so we can attach functionality to it later.
\begin{quote}\begin{description}
\item[{Returns}] \leavevmode
Reference to the clear button

\end{description}\end{quote}

\end{fulllineitems}

\index{del\_BtnConfirm() (src.Views.View\_TrackingScreen.TrackingWindow property)@\spxentry{del\_BtnConfirm()}\spxextra{src.Views.View\_TrackingScreen.TrackingWindow property}}

\begin{fulllineitems}
\phantomsection\label{\detokenize{index:src.Views.View_TrackingScreen.TrackingWindow.del_BtnConfirm}}\pysigline{\sphinxbfcode{\sphinxupquote{property }}\sphinxbfcode{\sphinxupquote{del\_BtnConfirm}}}
Getter for the Confirm button so we can attach functionality to it later.
\begin{quote}\begin{description}
\item[{Returns}] \leavevmode
Reference to the confirm button

\end{description}\end{quote}

\end{fulllineitems}

\index{del\_BtnStart() (src.Views.View\_TrackingScreen.TrackingWindow property)@\spxentry{del\_BtnStart()}\spxextra{src.Views.View\_TrackingScreen.TrackingWindow property}}

\begin{fulllineitems}
\phantomsection\label{\detokenize{index:src.Views.View_TrackingScreen.TrackingWindow.del_BtnStart}}\pysigline{\sphinxbfcode{\sphinxupquote{property }}\sphinxbfcode{\sphinxupquote{del\_BtnStart}}}
Getter for the Start button
\begin{quote}\begin{description}
\item[{Returns}] \leavevmode
Reference to the start button

\end{description}\end{quote}

\end{fulllineitems}

\index{del\_BtnStop() (src.Views.View\_TrackingScreen.TrackingWindow property)@\spxentry{del\_BtnStop()}\spxextra{src.Views.View\_TrackingScreen.TrackingWindow property}}

\begin{fulllineitems}
\phantomsection\label{\detokenize{index:src.Views.View_TrackingScreen.TrackingWindow.del_BtnStop}}\pysigline{\sphinxbfcode{\sphinxupquote{property }}\sphinxbfcode{\sphinxupquote{del\_BtnStop}}}
Getter for the Stop button
\begin{quote}\begin{description}
\item[{Returns}] \leavevmode
Reference to the stop button

\end{description}\end{quote}

\end{fulllineitems}

\index{del\_LblInstructor() (src.Views.View\_TrackingScreen.TrackingWindow property)@\spxentry{del\_LblInstructor()}\spxextra{src.Views.View\_TrackingScreen.TrackingWindow property}}

\begin{fulllineitems}
\phantomsection\label{\detokenize{index:src.Views.View_TrackingScreen.TrackingWindow.del_LblInstructor}}\pysigline{\sphinxbfcode{\sphinxupquote{property }}\sphinxbfcode{\sphinxupquote{del\_LblInstructor}}}
Getter for the Instructor label so we can attach functionality to it later.
\begin{quote}\begin{description}
\item[{Returns}] \leavevmode
The instructor label

\end{description}\end{quote}

\end{fulllineitems}

\index{del\_LblPilot() (src.Views.View\_TrackingScreen.TrackingWindow property)@\spxentry{del\_LblPilot()}\spxextra{src.Views.View\_TrackingScreen.TrackingWindow property}}

\begin{fulllineitems}
\phantomsection\label{\detokenize{index:src.Views.View_TrackingScreen.TrackingWindow.del_LblPilot}}\pysigline{\sphinxbfcode{\sphinxupquote{property }}\sphinxbfcode{\sphinxupquote{del\_LblPilot}}}
Getter for the Pilot label so we can attach functionality to it
\begin{quote}\begin{description}
\item[{Returns}] \leavevmode
The pilot label

\end{description}\end{quote}

\end{fulllineitems}

\index{del\_LblTimer() (src.Views.View\_TrackingScreen.TrackingWindow property)@\spxentry{del\_LblTimer()}\spxextra{src.Views.View\_TrackingScreen.TrackingWindow property}}

\begin{fulllineitems}
\phantomsection\label{\detokenize{index:src.Views.View_TrackingScreen.TrackingWindow.del_LblTimer}}\pysigline{\sphinxbfcode{\sphinxupquote{property }}\sphinxbfcode{\sphinxupquote{del\_LblTimer}}}
Getter property for the timer label. We need to attach a QTimer to it so it can count the time the
application has been tracking the drone.
\begin{quote}\begin{description}
\item[{Returns}] \leavevmode
The timer label

\end{description}\end{quote}

\end{fulllineitems}

\index{del\_TBInstructor() (src.Views.View\_TrackingScreen.TrackingWindow property)@\spxentry{del\_TBInstructor()}\spxextra{src.Views.View\_TrackingScreen.TrackingWindow property}}

\begin{fulllineitems}
\phantomsection\label{\detokenize{index:src.Views.View_TrackingScreen.TrackingWindow.del_TBInstructor}}\pysigline{\sphinxbfcode{\sphinxupquote{property }}\sphinxbfcode{\sphinxupquote{del\_TBInstructor}}}
Getter for the Instructor textbox so we can attach functionality to it later.
\begin{quote}\begin{description}
\item[{Returns}] \leavevmode
The instructor textbox

\end{description}\end{quote}

\end{fulllineitems}

\index{del\_TBPilot() (src.Views.View\_TrackingScreen.TrackingWindow property)@\spxentry{del\_TBPilot()}\spxextra{src.Views.View\_TrackingScreen.TrackingWindow property}}

\begin{fulllineitems}
\phantomsection\label{\detokenize{index:src.Views.View_TrackingScreen.TrackingWindow.del_TBPilot}}\pysigline{\sphinxbfcode{\sphinxupquote{property }}\sphinxbfcode{\sphinxupquote{del\_TBPilot}}}
Getter for the Pilot Textbox so we can attach functionality to it
\begin{quote}\begin{description}
\item[{Returns}] \leavevmode
The pilot textbox

\end{description}\end{quote}

\end{fulllineitems}

\index{del\_TEInstructions() (src.Views.View\_TrackingScreen.TrackingWindow property)@\spxentry{del\_TEInstructions()}\spxextra{src.Views.View\_TrackingScreen.TrackingWindow property}}

\begin{fulllineitems}
\phantomsection\label{\detokenize{index:src.Views.View_TrackingScreen.TrackingWindow.del_TEInstructions}}\pysigline{\sphinxbfcode{\sphinxupquote{property }}\sphinxbfcode{\sphinxupquote{del\_TEInstructions}}}
Getter for the instructions text edit box
\begin{quote}\begin{description}
\item[{Returns}] \leavevmode
Reference to the instructions text edit box

\end{description}\end{quote}

\end{fulllineitems}

\index{initView() (src.Views.View\_TrackingScreen.TrackingWindow method)@\spxentry{initView()}\spxextra{src.Views.View\_TrackingScreen.TrackingWindow method}}

\begin{fulllineitems}
\phantomsection\label{\detokenize{index:src.Views.View_TrackingScreen.TrackingWindow.initView}}\pysiglinewithargsret{\sphinxbfcode{\sphinxupquote{initView}}}{}{{ $\rightarrow$ None}}
Initializes and lays out all of the controls and elements on the view.
\begin{quote}\begin{description}
\item[{Returns}] \leavevmode
None

\end{description}\end{quote}

\end{fulllineitems}

\index{returnHome() (src.Views.View\_TrackingScreen.TrackingWindow method)@\spxentry{returnHome()}\spxextra{src.Views.View\_TrackingScreen.TrackingWindow method}}

\begin{fulllineitems}
\phantomsection\label{\detokenize{index:src.Views.View_TrackingScreen.TrackingWindow.returnHome}}\pysiglinewithargsret{\sphinxbfcode{\sphinxupquote{returnHome}}}{}{{ $\rightarrow$ None}}
Sends a signal to the main controller that the Return Home button was pushed.
\begin{quote}\begin{description}
\item[{Returns}] \leavevmode
none

\end{description}\end{quote}

\end{fulllineitems}

\index{setClrConfirmBtns() (src.Views.View\_TrackingScreen.TrackingWindow method)@\spxentry{setClrConfirmBtns()}\spxextra{src.Views.View\_TrackingScreen.TrackingWindow method}}

\begin{fulllineitems}
\phantomsection\label{\detokenize{index:src.Views.View_TrackingScreen.TrackingWindow.setClrConfirmBtns}}\pysiglinewithargsret{\sphinxbfcode{\sphinxupquote{setClrConfirmBtns}}}{}{{ $\rightarrow$ PyQt5.QtWidgets.QHBoxLayout}}
Sets the buttons for clearing and confirming the pilot, instructor, and flight instruction information.
\begin{quote}\begin{description}
\item[{Returns}] \leavevmode
The confirmation button

\end{description}\end{quote}

\end{fulllineitems}

\index{setFlightInstructions() (src.Views.View\_TrackingScreen.TrackingWindow method)@\spxentry{setFlightInstructions()}\spxextra{src.Views.View\_TrackingScreen.TrackingWindow method}}

\begin{fulllineitems}
\phantomsection\label{\detokenize{index:src.Views.View_TrackingScreen.TrackingWindow.setFlightInstructions}}\pysiglinewithargsret{\sphinxbfcode{\sphinxupquote{setFlightInstructions}}}{}{{ $\rightarrow$ PyQt5.QtWidgets.QVBoxLayout}}
Sets the textbox that will allow the instructor to type in the flight instructions for the pilot
to try to match.
\begin{quote}\begin{description}
\item[{Returns}] \leavevmode
A vertical layout with the Instructions label on top of the text box

\end{description}\end{quote}

\end{fulllineitems}

\index{setInstructor() (src.Views.View\_TrackingScreen.TrackingWindow method)@\spxentry{setInstructor()}\spxextra{src.Views.View\_TrackingScreen.TrackingWindow method}}

\begin{fulllineitems}
\phantomsection\label{\detokenize{index:src.Views.View_TrackingScreen.TrackingWindow.setInstructor}}\pysiglinewithargsret{\sphinxbfcode{\sphinxupquote{setInstructor}}}{}{{ $\rightarrow$ PyQt5.QtWidgets.QVBoxLayout}}
Sets up the instructor label and the textbox that will be used to set the instructor flying during this
session.
\begin{quote}\begin{description}
\item[{Returns}] \leavevmode
Returns a vertical layout with the instructor label over the instructor textbox

\end{description}\end{quote}

\end{fulllineitems}

\index{setPilot() (src.Views.View\_TrackingScreen.TrackingWindow method)@\spxentry{setPilot()}\spxextra{src.Views.View\_TrackingScreen.TrackingWindow method}}

\begin{fulllineitems}
\phantomsection\label{\detokenize{index:src.Views.View_TrackingScreen.TrackingWindow.setPilot}}\pysiglinewithargsret{\sphinxbfcode{\sphinxupquote{setPilot}}}{}{{ $\rightarrow$ PyQt5.QtWidgets.QVBoxLayout}}
Sets up the Pilot label and the textbox that will be used to set the pilot flying during this
session.
\begin{quote}\begin{description}
\item[{Returns}] \leavevmode
Returns a vertical layout with the pilot label over the pilot textbox

\end{description}\end{quote}

\end{fulllineitems}

\index{setStartAndStopBtns() (src.Views.View\_TrackingScreen.TrackingWindow method)@\spxentry{setStartAndStopBtns()}\spxextra{src.Views.View\_TrackingScreen.TrackingWindow method}}

\begin{fulllineitems}
\phantomsection\label{\detokenize{index:src.Views.View_TrackingScreen.TrackingWindow.setStartAndStopBtns}}\pysiglinewithargsret{\sphinxbfcode{\sphinxupquote{setStartAndStopBtns}}}{}{{ $\rightarrow$ PyQt5.QtWidgets.QHBoxLayout}}
Sets up the start and stop buttons for tracking the drones.
\begin{quote}\begin{description}
\item[{Returns}] \leavevmode


\end{description}\end{quote}

\end{fulllineitems}

\index{setStatusLabel() (src.Views.View\_TrackingScreen.TrackingWindow method)@\spxentry{setStatusLabel()}\spxextra{src.Views.View\_TrackingScreen.TrackingWindow method}}

\begin{fulllineitems}
\phantomsection\label{\detokenize{index:src.Views.View_TrackingScreen.TrackingWindow.setStatusLabel}}\pysiglinewithargsret{\sphinxbfcode{\sphinxupquote{setStatusLabel}}}{\emph{text}}{{ $\rightarrow$ PyQt5.QtWidgets.QLabel}}
Sets up a status label for the window
\begin{quote}\begin{description}
\item[{Returns}] \leavevmode
Label of the application taken from the “text” parameter

\end{description}\end{quote}

\end{fulllineitems}

\index{setSubTitle() (src.Views.View\_TrackingScreen.TrackingWindow method)@\spxentry{setSubTitle()}\spxextra{src.Views.View\_TrackingScreen.TrackingWindow method}}

\begin{fulllineitems}
\phantomsection\label{\detokenize{index:src.Views.View_TrackingScreen.TrackingWindow.setSubTitle}}\pysiglinewithargsret{\sphinxbfcode{\sphinxupquote{setSubTitle}}}{\emph{text}}{{ $\rightarrow$ PyQt5.QtWidgets.QLabel}}
Sets up a subtitle label for the window
\begin{quote}\begin{description}
\item[{Returns}] \leavevmode
Subtitle of the application taken from the “text” parameter

\end{description}\end{quote}

\end{fulllineitems}

\index{setTimerLabel() (src.Views.View\_TrackingScreen.TrackingWindow method)@\spxentry{setTimerLabel()}\spxextra{src.Views.View\_TrackingScreen.TrackingWindow method}}

\begin{fulllineitems}
\phantomsection\label{\detokenize{index:src.Views.View_TrackingScreen.TrackingWindow.setTimerLabel}}\pysiglinewithargsret{\sphinxbfcode{\sphinxupquote{setTimerLabel}}}{}{{ $\rightarrow$ PyQt5.QtWidgets.QVBoxLayout}}
Sets the label that will continuously update and display the time that the application has been
actively tracking. Need to attach a QTimer() to it.
\begin{quote}\begin{description}
\item[{Returns}] \leavevmode
The label that will contain the timer.

\end{description}\end{quote}

\end{fulllineitems}

\index{setTitle() (src.Views.View\_TrackingScreen.TrackingWindow method)@\spxentry{setTitle()}\spxextra{src.Views.View\_TrackingScreen.TrackingWindow method}}

\begin{fulllineitems}
\phantomsection\label{\detokenize{index:src.Views.View_TrackingScreen.TrackingWindow.setTitle}}\pysiglinewithargsret{\sphinxbfcode{\sphinxupquote{setTitle}}}{}{{ $\rightarrow$ PyQt5.QtWidgets.QVBoxLayout}}
Sets up the title with the application title on top and the name of the screen just below it.
\begin{quote}\begin{description}
\item[{Returns}] \leavevmode
Layout with the application title and screen title labels

\end{description}\end{quote}

\end{fulllineitems}

\index{set\_BtnClear() (src.Views.View\_TrackingScreen.TrackingWindow property)@\spxentry{set\_BtnClear()}\spxextra{src.Views.View\_TrackingScreen.TrackingWindow property}}

\begin{fulllineitems}
\phantomsection\label{\detokenize{index:src.Views.View_TrackingScreen.TrackingWindow.set_BtnClear}}\pysigline{\sphinxbfcode{\sphinxupquote{property }}\sphinxbfcode{\sphinxupquote{set\_BtnClear}}}
Getter for the Clear button so we can attach functionality to it later.
\begin{quote}\begin{description}
\item[{Returns}] \leavevmode
Reference to the clear button

\end{description}\end{quote}

\end{fulllineitems}

\index{set\_BtnConfirm() (src.Views.View\_TrackingScreen.TrackingWindow property)@\spxentry{set\_BtnConfirm()}\spxextra{src.Views.View\_TrackingScreen.TrackingWindow property}}

\begin{fulllineitems}
\phantomsection\label{\detokenize{index:src.Views.View_TrackingScreen.TrackingWindow.set_BtnConfirm}}\pysigline{\sphinxbfcode{\sphinxupquote{property }}\sphinxbfcode{\sphinxupquote{set\_BtnConfirm}}}
Getter for the Confirm button so we can attach functionality to it later.
\begin{quote}\begin{description}
\item[{Returns}] \leavevmode
Reference to the confirm button

\end{description}\end{quote}

\end{fulllineitems}

\index{set\_BtnStart() (src.Views.View\_TrackingScreen.TrackingWindow property)@\spxentry{set\_BtnStart()}\spxextra{src.Views.View\_TrackingScreen.TrackingWindow property}}

\begin{fulllineitems}
\phantomsection\label{\detokenize{index:src.Views.View_TrackingScreen.TrackingWindow.set_BtnStart}}\pysigline{\sphinxbfcode{\sphinxupquote{property }}\sphinxbfcode{\sphinxupquote{set\_BtnStart}}}
Getter for the Start button
\begin{quote}\begin{description}
\item[{Returns}] \leavevmode
Reference to the start button

\end{description}\end{quote}

\end{fulllineitems}

\index{set\_BtnStop() (src.Views.View\_TrackingScreen.TrackingWindow property)@\spxentry{set\_BtnStop()}\spxextra{src.Views.View\_TrackingScreen.TrackingWindow property}}

\begin{fulllineitems}
\phantomsection\label{\detokenize{index:src.Views.View_TrackingScreen.TrackingWindow.set_BtnStop}}\pysigline{\sphinxbfcode{\sphinxupquote{property }}\sphinxbfcode{\sphinxupquote{set\_BtnStop}}}
Getter for the Stop button
\begin{quote}\begin{description}
\item[{Returns}] \leavevmode
Reference to the stop button

\end{description}\end{quote}

\end{fulllineitems}

\index{set\_LblInstructor() (src.Views.View\_TrackingScreen.TrackingWindow property)@\spxentry{set\_LblInstructor()}\spxextra{src.Views.View\_TrackingScreen.TrackingWindow property}}

\begin{fulllineitems}
\phantomsection\label{\detokenize{index:src.Views.View_TrackingScreen.TrackingWindow.set_LblInstructor}}\pysigline{\sphinxbfcode{\sphinxupquote{property }}\sphinxbfcode{\sphinxupquote{set\_LblInstructor}}}
Getter for the Instructor label so we can attach functionality to it later.
\begin{quote}\begin{description}
\item[{Returns}] \leavevmode
The instructor label

\end{description}\end{quote}

\end{fulllineitems}

\index{set\_LblPilot() (src.Views.View\_TrackingScreen.TrackingWindow property)@\spxentry{set\_LblPilot()}\spxextra{src.Views.View\_TrackingScreen.TrackingWindow property}}

\begin{fulllineitems}
\phantomsection\label{\detokenize{index:src.Views.View_TrackingScreen.TrackingWindow.set_LblPilot}}\pysigline{\sphinxbfcode{\sphinxupquote{property }}\sphinxbfcode{\sphinxupquote{set\_LblPilot}}}
Getter for the Pilot label so we can attach functionality to it
\begin{quote}\begin{description}
\item[{Returns}] \leavevmode
The pilot label

\end{description}\end{quote}

\end{fulllineitems}

\index{set\_LblTimer() (src.Views.View\_TrackingScreen.TrackingWindow property)@\spxentry{set\_LblTimer()}\spxextra{src.Views.View\_TrackingScreen.TrackingWindow property}}

\begin{fulllineitems}
\phantomsection\label{\detokenize{index:src.Views.View_TrackingScreen.TrackingWindow.set_LblTimer}}\pysigline{\sphinxbfcode{\sphinxupquote{property }}\sphinxbfcode{\sphinxupquote{set\_LblTimer}}}
Getter property for the timer label. We need to attach a QTimer to it so it can count the time the
application has been tracking the drone.
\begin{quote}\begin{description}
\item[{Returns}] \leavevmode
The timer label

\end{description}\end{quote}

\end{fulllineitems}

\index{set\_TBInstructor() (src.Views.View\_TrackingScreen.TrackingWindow property)@\spxentry{set\_TBInstructor()}\spxextra{src.Views.View\_TrackingScreen.TrackingWindow property}}

\begin{fulllineitems}
\phantomsection\label{\detokenize{index:src.Views.View_TrackingScreen.TrackingWindow.set_TBInstructor}}\pysigline{\sphinxbfcode{\sphinxupquote{property }}\sphinxbfcode{\sphinxupquote{set\_TBInstructor}}}
Getter for the Instructor textbox so we can attach functionality to it later.
\begin{quote}\begin{description}
\item[{Returns}] \leavevmode
The instructor textbox

\end{description}\end{quote}

\end{fulllineitems}

\index{set\_TBPilot() (src.Views.View\_TrackingScreen.TrackingWindow property)@\spxentry{set\_TBPilot()}\spxextra{src.Views.View\_TrackingScreen.TrackingWindow property}}

\begin{fulllineitems}
\phantomsection\label{\detokenize{index:src.Views.View_TrackingScreen.TrackingWindow.set_TBPilot}}\pysigline{\sphinxbfcode{\sphinxupquote{property }}\sphinxbfcode{\sphinxupquote{set\_TBPilot}}}
Getter for the Pilot Textbox so we can attach functionality to it
\begin{quote}\begin{description}
\item[{Returns}] \leavevmode
The pilot textbox

\end{description}\end{quote}

\end{fulllineitems}

\index{set\_TEInstructions() (src.Views.View\_TrackingScreen.TrackingWindow property)@\spxentry{set\_TEInstructions()}\spxextra{src.Views.View\_TrackingScreen.TrackingWindow property}}

\begin{fulllineitems}
\phantomsection\label{\detokenize{index:src.Views.View_TrackingScreen.TrackingWindow.set_TEInstructions}}\pysigline{\sphinxbfcode{\sphinxupquote{property }}\sphinxbfcode{\sphinxupquote{set\_TEInstructions}}}
Getter for the instructions text edit box
\begin{quote}\begin{description}
\item[{Returns}] \leavevmode
Reference to the instructions text edit box

\end{description}\end{quote}

\end{fulllineitems}

\index{startTracking() (src.Views.View\_TrackingScreen.TrackingWindow method)@\spxentry{startTracking()}\spxextra{src.Views.View\_TrackingScreen.TrackingWindow method}}

\begin{fulllineitems}
\phantomsection\label{\detokenize{index:src.Views.View_TrackingScreen.TrackingWindow.startTracking}}\pysiglinewithargsret{\sphinxbfcode{\sphinxupquote{startTracking}}}{}{{ $\rightarrow$ None}}
Sends a signal to the main controller that the Start Tracking button was pushed.
\begin{quote}\begin{description}
\item[{Returns}] \leavevmode
none

\end{description}\end{quote}

\end{fulllineitems}

\index{stopTracking() (src.Views.View\_TrackingScreen.TrackingWindow method)@\spxentry{stopTracking()}\spxextra{src.Views.View\_TrackingScreen.TrackingWindow method}}

\begin{fulllineitems}
\phantomsection\label{\detokenize{index:src.Views.View_TrackingScreen.TrackingWindow.stopTracking}}\pysiglinewithargsret{\sphinxbfcode{\sphinxupquote{stopTracking}}}{}{{ $\rightarrow$ None}}
Sends a signal to the main controller that the Stop Tracking button was pushed.
\begin{quote}\begin{description}
\item[{Returns}] \leavevmode
none

\end{description}\end{quote}

\end{fulllineitems}


\end{fulllineitems}



\chapter{Verify Setup Screen}
\label{\detokenize{index:module-src.Views.View_VerifySetupScreen}}\label{\detokenize{index:verify-setup-screen}}\index{src.Views.View\_VerifySetupScreen (module)@\spxentry{src.Views.View\_VerifySetupScreen}\spxextra{module}}\index{VerifySetupWindow (class in src.Views.View\_VerifySetupScreen)@\spxentry{VerifySetupWindow}\spxextra{class in src.Views.View\_VerifySetupScreen}}

\begin{fulllineitems}
\phantomsection\label{\detokenize{index:src.Views.View_VerifySetupScreen.VerifySetupWindow}}\pysiglinewithargsret{\sphinxbfcode{\sphinxupquote{class }}\sphinxcode{\sphinxupquote{src.Views.View\_VerifySetupScreen.}}\sphinxbfcode{\sphinxupquote{VerifySetupWindow}}}{\emph{phoneControl: Controllers.PhoneController.PhoneControl}}{}
The view for the verify setup page that is shown when the user presses the “Verify Setup” button on the home page.
\begin{quote}\begin{description}
\item[{Variables}] \leavevmode\begin{itemize}
\item {} 
\sphinxstyleliteralstrong{\sphinxupquote{\_\_btnPhoneSync}} \textendash{} The class property for the ‘Phone Sync’ button.

\item {} 
\sphinxstyleliteralstrong{\sphinxupquote{\_\_btnTestLight}} \textendash{} The class property for the ‘Test Light’ button.

\item {} 
\sphinxstyleliteralstrong{\sphinxupquote{\_\_btnTestFull}} \textendash{} The class property for the ‘Test Full Setup’ button.

\item {} 
\sphinxstyleliteralstrong{\sphinxupquote{\_\_btnCheck}} \textendash{} The class property for the ‘Check Status’ button.

\item {} 
\sphinxstyleliteralstrong{\sphinxupquote{\_\_btnHome}} \textendash{} The class property for the ‘Return to Home’ button.

\end{itemize}

\end{description}\end{quote}
\index{BtnCheck() (src.Views.View\_VerifySetupScreen.VerifySetupWindow property)@\spxentry{BtnCheck()}\spxextra{src.Views.View\_VerifySetupScreen.VerifySetupWindow property}}

\begin{fulllineitems}
\phantomsection\label{\detokenize{index:src.Views.View_VerifySetupScreen.VerifySetupWindow.BtnCheck}}\pysigline{\sphinxbfcode{\sphinxupquote{property }}\sphinxbfcode{\sphinxupquote{BtnCheck}}}
The check status button for the view so we can attach functionality to it later on.
Is used to check the status of the set up procedures.
\begin{quote}\begin{description}
\item[{Returns}] \leavevmode
None

\end{description}\end{quote}

\end{fulllineitems}

\index{BtnHome() (src.Views.View\_VerifySetupScreen.VerifySetupWindow property)@\spxentry{BtnHome()}\spxextra{src.Views.View\_VerifySetupScreen.VerifySetupWindow property}}

\begin{fulllineitems}
\phantomsection\label{\detokenize{index:src.Views.View_VerifySetupScreen.VerifySetupWindow.BtnHome}}\pysigline{\sphinxbfcode{\sphinxupquote{property }}\sphinxbfcode{\sphinxupquote{BtnHome}}}
The home for the view. Is used to return to home screen.
\begin{quote}\begin{description}
\item[{Returns}] \leavevmode
None

\end{description}\end{quote}

\end{fulllineitems}

\index{BtnPhoneSync() (src.Views.View\_VerifySetupScreen.VerifySetupWindow property)@\spxentry{BtnPhoneSync()}\spxextra{src.Views.View\_VerifySetupScreen.VerifySetupWindow property}}

\begin{fulllineitems}
\phantomsection\label{\detokenize{index:src.Views.View_VerifySetupScreen.VerifySetupWindow.BtnPhoneSync}}\pysigline{\sphinxbfcode{\sphinxupquote{property }}\sphinxbfcode{\sphinxupquote{BtnPhoneSync}}}
The phone sync button so we can attach functionality to it later on.
\begin{quote}\begin{description}
\item[{Returns}] \leavevmode
The reference to the phoneSync button

\end{description}\end{quote}

\end{fulllineitems}

\index{BtnTestFull() (src.Views.View\_VerifySetupScreen.VerifySetupWindow property)@\spxentry{BtnTestFull()}\spxextra{src.Views.View\_VerifySetupScreen.VerifySetupWindow property}}

\begin{fulllineitems}
\phantomsection\label{\detokenize{index:src.Views.View_VerifySetupScreen.VerifySetupWindow.BtnTestFull}}\pysigline{\sphinxbfcode{\sphinxupquote{property }}\sphinxbfcode{\sphinxupquote{BtnTestFull}}}
The test full setup button for the view so we can attach functionality to it later on.
Is used to import past flight files.
\begin{quote}\begin{description}
\item[{Returns}] \leavevmode
None

\end{description}\end{quote}

\end{fulllineitems}

\index{BtnTestLight() (src.Views.View\_VerifySetupScreen.VerifySetupWindow property)@\spxentry{BtnTestLight()}\spxextra{src.Views.View\_VerifySetupScreen.VerifySetupWindow property}}

\begin{fulllineitems}
\phantomsection\label{\detokenize{index:src.Views.View_VerifySetupScreen.VerifySetupWindow.BtnTestLight}}\pysigline{\sphinxbfcode{\sphinxupquote{property }}\sphinxbfcode{\sphinxupquote{BtnTestLight}}}
The test light button so we can attach functionality to it later on.
\begin{quote}\begin{description}
\item[{Returns}] \leavevmode
None

\end{description}\end{quote}

\end{fulllineitems}

\index{checkStatus() (src.Views.View\_VerifySetupScreen.VerifySetupWindow method)@\spxentry{checkStatus()}\spxextra{src.Views.View\_VerifySetupScreen.VerifySetupWindow method}}

\begin{fulllineitems}
\phantomsection\label{\detokenize{index:src.Views.View_VerifySetupScreen.VerifySetupWindow.checkStatus}}\pysiglinewithargsret{\sphinxbfcode{\sphinxupquote{checkStatus}}}{}{{ $\rightarrow$ None}}
Shows the status of the system setup.
\begin{quote}\begin{description}
\item[{Returns}] \leavevmode
None

\end{description}\end{quote}

\end{fulllineitems}

\index{del\_BtnCheck() (src.Views.View\_VerifySetupScreen.VerifySetupWindow property)@\spxentry{del\_BtnCheck()}\spxextra{src.Views.View\_VerifySetupScreen.VerifySetupWindow property}}

\begin{fulllineitems}
\phantomsection\label{\detokenize{index:src.Views.View_VerifySetupScreen.VerifySetupWindow.del_BtnCheck}}\pysigline{\sphinxbfcode{\sphinxupquote{property }}\sphinxbfcode{\sphinxupquote{del\_BtnCheck}}}
The check status button for the view so we can attach functionality to it later on.
Is used to check the status of the set up procedures.
\begin{quote}\begin{description}
\item[{Returns}] \leavevmode
None

\end{description}\end{quote}

\end{fulllineitems}

\index{del\_BtnHome() (src.Views.View\_VerifySetupScreen.VerifySetupWindow property)@\spxentry{del\_BtnHome()}\spxextra{src.Views.View\_VerifySetupScreen.VerifySetupWindow property}}

\begin{fulllineitems}
\phantomsection\label{\detokenize{index:src.Views.View_VerifySetupScreen.VerifySetupWindow.del_BtnHome}}\pysigline{\sphinxbfcode{\sphinxupquote{property }}\sphinxbfcode{\sphinxupquote{del\_BtnHome}}}
The home for the view. Is used to return to home screen.
\begin{quote}\begin{description}
\item[{Returns}] \leavevmode
None

\end{description}\end{quote}

\end{fulllineitems}

\index{del\_BtnPhoneSync() (src.Views.View\_VerifySetupScreen.VerifySetupWindow property)@\spxentry{del\_BtnPhoneSync()}\spxextra{src.Views.View\_VerifySetupScreen.VerifySetupWindow property}}

\begin{fulllineitems}
\phantomsection\label{\detokenize{index:src.Views.View_VerifySetupScreen.VerifySetupWindow.del_BtnPhoneSync}}\pysigline{\sphinxbfcode{\sphinxupquote{property }}\sphinxbfcode{\sphinxupquote{del\_BtnPhoneSync}}}
The phone sync button so we can attach functionality to it later on.
\begin{quote}\begin{description}
\item[{Returns}] \leavevmode
The reference to the phoneSync button

\end{description}\end{quote}

\end{fulllineitems}

\index{del\_BtnTestFull() (src.Views.View\_VerifySetupScreen.VerifySetupWindow property)@\spxentry{del\_BtnTestFull()}\spxextra{src.Views.View\_VerifySetupScreen.VerifySetupWindow property}}

\begin{fulllineitems}
\phantomsection\label{\detokenize{index:src.Views.View_VerifySetupScreen.VerifySetupWindow.del_BtnTestFull}}\pysigline{\sphinxbfcode{\sphinxupquote{property }}\sphinxbfcode{\sphinxupquote{del\_BtnTestFull}}}
The test full setup button for the view so we can attach functionality to it later on.
Is used to import past flight files.
\begin{quote}\begin{description}
\item[{Returns}] \leavevmode
None

\end{description}\end{quote}

\end{fulllineitems}

\index{del\_BtnTestLight() (src.Views.View\_VerifySetupScreen.VerifySetupWindow property)@\spxentry{del\_BtnTestLight()}\spxextra{src.Views.View\_VerifySetupScreen.VerifySetupWindow property}}

\begin{fulllineitems}
\phantomsection\label{\detokenize{index:src.Views.View_VerifySetupScreen.VerifySetupWindow.del_BtnTestLight}}\pysigline{\sphinxbfcode{\sphinxupquote{property }}\sphinxbfcode{\sphinxupquote{del\_BtnTestLight}}}
The test light button so we can attach functionality to it later on.
\begin{quote}\begin{description}
\item[{Returns}] \leavevmode
None

\end{description}\end{quote}

\end{fulllineitems}

\index{initView() (src.Views.View\_VerifySetupScreen.VerifySetupWindow method)@\spxentry{initView()}\spxextra{src.Views.View\_VerifySetupScreen.VerifySetupWindow method}}

\begin{fulllineitems}
\phantomsection\label{\detokenize{index:src.Views.View_VerifySetupScreen.VerifySetupWindow.initView}}\pysiglinewithargsret{\sphinxbfcode{\sphinxupquote{initView}}}{}{{ $\rightarrow$ None}}
Sets up the view and lays out all of the components.
\begin{quote}\begin{description}
\item[{Returns}] \leavevmode
None

\end{description}\end{quote}

\end{fulllineitems}

\index{returnHome() (src.Views.View\_VerifySetupScreen.VerifySetupWindow method)@\spxentry{returnHome()}\spxextra{src.Views.View\_VerifySetupScreen.VerifySetupWindow method}}

\begin{fulllineitems}
\phantomsection\label{\detokenize{index:src.Views.View_VerifySetupScreen.VerifySetupWindow.returnHome}}\pysiglinewithargsret{\sphinxbfcode{\sphinxupquote{returnHome}}}{}{{ $\rightarrow$ None}}
Sends a signal to the main controller that the Return Home button was pushed.
\begin{quote}\begin{description}
\item[{Returns}] \leavevmode
none

\end{description}\end{quote}

\end{fulllineitems}

\index{setButtonLayout() (src.Views.View\_VerifySetupScreen.VerifySetupWindow method)@\spxentry{setButtonLayout()}\spxextra{src.Views.View\_VerifySetupScreen.VerifySetupWindow method}}

\begin{fulllineitems}
\phantomsection\label{\detokenize{index:src.Views.View_VerifySetupScreen.VerifySetupWindow.setButtonLayout}}\pysiglinewithargsret{\sphinxbfcode{\sphinxupquote{setButtonLayout}}}{}{{ $\rightarrow$ PyQt5.QtWidgets.QHBoxLayout}}
Lays out the ‘Phone Sync’, ‘Test Light’ and ‘Test Full Setup’ buttons into a horizontal layout to be
put on screen.
\begin{quote}\begin{description}
\item[{Returns}] \leavevmode
The horizontal layout containing the 3 buttons

\end{description}\end{quote}

\end{fulllineitems}

\index{setTitle() (src.Views.View\_VerifySetupScreen.VerifySetupWindow method)@\spxentry{setTitle()}\spxextra{src.Views.View\_VerifySetupScreen.VerifySetupWindow method}}

\begin{fulllineitems}
\phantomsection\label{\detokenize{index:src.Views.View_VerifySetupScreen.VerifySetupWindow.setTitle}}\pysiglinewithargsret{\sphinxbfcode{\sphinxupquote{setTitle}}}{}{{ $\rightarrow$ PyQt5.QtWidgets.QVBoxLayout}}
Sets up the title with the application title on top and the name of the screen just below it.
:return: Layout with the application title and screen title labels

\end{fulllineitems}

\index{set\_BtnCheck() (src.Views.View\_VerifySetupScreen.VerifySetupWindow property)@\spxentry{set\_BtnCheck()}\spxextra{src.Views.View\_VerifySetupScreen.VerifySetupWindow property}}

\begin{fulllineitems}
\phantomsection\label{\detokenize{index:src.Views.View_VerifySetupScreen.VerifySetupWindow.set_BtnCheck}}\pysigline{\sphinxbfcode{\sphinxupquote{property }}\sphinxbfcode{\sphinxupquote{set\_BtnCheck}}}
The check status button for the view so we can attach functionality to it later on.
Is used to check the status of the set up procedures.
\begin{quote}\begin{description}
\item[{Returns}] \leavevmode
None

\end{description}\end{quote}

\end{fulllineitems}

\index{set\_BtnHome() (src.Views.View\_VerifySetupScreen.VerifySetupWindow property)@\spxentry{set\_BtnHome()}\spxextra{src.Views.View\_VerifySetupScreen.VerifySetupWindow property}}

\begin{fulllineitems}
\phantomsection\label{\detokenize{index:src.Views.View_VerifySetupScreen.VerifySetupWindow.set_BtnHome}}\pysigline{\sphinxbfcode{\sphinxupquote{property }}\sphinxbfcode{\sphinxupquote{set\_BtnHome}}}
The home for the view. Is used to return to home screen.
\begin{quote}\begin{description}
\item[{Returns}] \leavevmode
None

\end{description}\end{quote}

\end{fulllineitems}

\index{set\_BtnPhoneSync() (src.Views.View\_VerifySetupScreen.VerifySetupWindow property)@\spxentry{set\_BtnPhoneSync()}\spxextra{src.Views.View\_VerifySetupScreen.VerifySetupWindow property}}

\begin{fulllineitems}
\phantomsection\label{\detokenize{index:src.Views.View_VerifySetupScreen.VerifySetupWindow.set_BtnPhoneSync}}\pysigline{\sphinxbfcode{\sphinxupquote{property }}\sphinxbfcode{\sphinxupquote{set\_BtnPhoneSync}}}
The phone sync button so we can attach functionality to it later on.
\begin{quote}\begin{description}
\item[{Returns}] \leavevmode
The reference to the phoneSync button

\end{description}\end{quote}

\end{fulllineitems}

\index{set\_BtnTestFull() (src.Views.View\_VerifySetupScreen.VerifySetupWindow property)@\spxentry{set\_BtnTestFull()}\spxextra{src.Views.View\_VerifySetupScreen.VerifySetupWindow property}}

\begin{fulllineitems}
\phantomsection\label{\detokenize{index:src.Views.View_VerifySetupScreen.VerifySetupWindow.set_BtnTestFull}}\pysigline{\sphinxbfcode{\sphinxupquote{property }}\sphinxbfcode{\sphinxupquote{set\_BtnTestFull}}}
The test full setup button for the view so we can attach functionality to it later on.
Is used to import past flight files.
\begin{quote}\begin{description}
\item[{Returns}] \leavevmode
None

\end{description}\end{quote}

\end{fulllineitems}

\index{set\_BtnTestLight() (src.Views.View\_VerifySetupScreen.VerifySetupWindow property)@\spxentry{set\_BtnTestLight()}\spxextra{src.Views.View\_VerifySetupScreen.VerifySetupWindow property}}

\begin{fulllineitems}
\phantomsection\label{\detokenize{index:src.Views.View_VerifySetupScreen.VerifySetupWindow.set_BtnTestLight}}\pysigline{\sphinxbfcode{\sphinxupquote{property }}\sphinxbfcode{\sphinxupquote{set\_BtnTestLight}}}
The test light button so we can attach functionality to it later on.
\begin{quote}\begin{description}
\item[{Returns}] \leavevmode
None

\end{description}\end{quote}

\end{fulllineitems}

\index{setupPicture() (src.Views.View\_VerifySetupScreen.VerifySetupWindow method)@\spxentry{setupPicture()}\spxextra{src.Views.View\_VerifySetupScreen.VerifySetupWindow method}}

\begin{fulllineitems}
\phantomsection\label{\detokenize{index:src.Views.View_VerifySetupScreen.VerifySetupWindow.setupPicture}}\pysiglinewithargsret{\sphinxbfcode{\sphinxupquote{setupPicture}}}{}{{ $\rightarrow$ PyQt5.QtWidgets.QLabel}}
Used for configuring the display for the logo on the screen.
\begin{quote}\begin{description}
\item[{Returns}] \leavevmode
Returns the label that contains our logo so it can be put on the main panel.

\end{description}\end{quote}

\end{fulllineitems}

\index{syncPhone() (src.Views.View\_VerifySetupScreen.VerifySetupWindow method)@\spxentry{syncPhone()}\spxextra{src.Views.View\_VerifySetupScreen.VerifySetupWindow method}}

\begin{fulllineitems}
\phantomsection\label{\detokenize{index:src.Views.View_VerifySetupScreen.VerifySetupWindow.syncPhone}}\pysiglinewithargsret{\sphinxbfcode{\sphinxupquote{syncPhone}}}{}{{ $\rightarrow$ None}}
Runs phone sync test.
\begin{quote}\begin{description}
\item[{Returns}] \leavevmode
None

\end{description}\end{quote}

\end{fulllineitems}

\index{testFull() (src.Views.View\_VerifySetupScreen.VerifySetupWindow method)@\spxentry{testFull()}\spxextra{src.Views.View\_VerifySetupScreen.VerifySetupWindow method}}

\begin{fulllineitems}
\phantomsection\label{\detokenize{index:src.Views.View_VerifySetupScreen.VerifySetupWindow.testFull}}\pysiglinewithargsret{\sphinxbfcode{\sphinxupquote{testFull}}}{}{{ $\rightarrow$ None}}
Runs full system test.
\begin{quote}\begin{description}
\item[{Returns}] \leavevmode
None

\end{description}\end{quote}

\end{fulllineitems}

\index{testLight() (src.Views.View\_VerifySetupScreen.VerifySetupWindow method)@\spxentry{testLight()}\spxextra{src.Views.View\_VerifySetupScreen.VerifySetupWindow method}}

\begin{fulllineitems}
\phantomsection\label{\detokenize{index:src.Views.View_VerifySetupScreen.VerifySetupWindow.testLight}}\pysiglinewithargsret{\sphinxbfcode{\sphinxupquote{testLight}}}{}{{ $\rightarrow$ None}}
Runs light test.
\begin{quote}\begin{description}
\item[{Returns}] \leavevmode
None

\end{description}\end{quote}

\end{fulllineitems}


\end{fulllineitems}



\chapter{MatPlot Graph}
\label{\detokenize{index:module-src.Views.Graph}}\label{\detokenize{index:matplot-graph}}\index{src.Views.Graph (module)@\spxentry{src.Views.Graph}\spxextra{module}}\index{checkCoordinates() (in module src.Views.Graph)@\spxentry{checkCoordinates()}\spxextra{in module src.Views.Graph}}

\begin{fulllineitems}
\phantomsection\label{\detokenize{index:src.Views.Graph.checkCoordinates}}\pysiglinewithargsret{\sphinxcode{\sphinxupquote{src.Views.Graph.}}\sphinxbfcode{\sphinxupquote{checkCoordinates}}}{\emph{flightDict: dict}}{}
Reads the input dictionary of flight data from a .flight file.
\begin{quote}\begin{description}
\item[{Parameters}] \leavevmode
\sphinxstyleliteralstrong{\sphinxupquote{flightDict}} \textendash{} Dictionary containing flight data.

\item[{Returns}] \leavevmode
Dictionary of flight data.

\end{description}\end{quote}

\end{fulllineitems}

\index{checkLegalInput() (in module src.Views.Graph)@\spxentry{checkLegalInput()}\spxextra{in module src.Views.Graph}}

\begin{fulllineitems}
\phantomsection\label{\detokenize{index:src.Views.Graph.checkLegalInput}}\pysiglinewithargsret{\sphinxcode{\sphinxupquote{src.Views.Graph.}}\sphinxbfcode{\sphinxupquote{checkLegalInput}}}{\emph{x}, \emph{y}, \emph{z}}{}
Checks if the inputted 3D coordinate is within legal bounds. Legal bounds are:
x, y, z \textgreater{} 0 and x \textless{} 15 and y \textless{} 15 and z \textless{} 10.
\begin{quote}\begin{description}
\item[{Parameters}] \leavevmode\begin{itemize}
\item {} 
\sphinxstyleliteralstrong{\sphinxupquote{x}} \textendash{} x value to check.

\item {} 
\sphinxstyleliteralstrong{\sphinxupquote{y}} \textendash{} y value to check.

\item {} 
\sphinxstyleliteralstrong{\sphinxupquote{z}} \textendash{} z value to check.

\end{itemize}

\item[{Returns}] \leavevmode
A boolean denoting if legal or not (true if legal, false if outside bounds).

\end{description}\end{quote}

\end{fulllineitems}

\index{computeVelocity() (in module src.Views.Graph)@\spxentry{computeVelocity()}\spxextra{in module src.Views.Graph}}

\begin{fulllineitems}
\phantomsection\label{\detokenize{index:src.Views.Graph.computeVelocity}}\pysiglinewithargsret{\sphinxcode{\sphinxupquote{src.Views.Graph.}}\sphinxbfcode{\sphinxupquote{computeVelocity}}}{\emph{x1}, \emph{y1}, \emph{z1}, \emph{x2}, \emph{y2}, \emph{z2}, \emph{t1}, \emph{t2}}{}
Computes the velocity of the drone between two points (x1,y1,z1) and (x2,y2,z2) at respective times t1 and t2.
\begin{quote}\begin{description}
\item[{Returns}] \leavevmode
A float value representing the velocity of the drone.

\end{description}\end{quote}

\end{fulllineitems}

\index{computeVelocityStatistics() (in module src.Views.Graph)@\spxentry{computeVelocityStatistics()}\spxextra{in module src.Views.Graph}}

\begin{fulllineitems}
\phantomsection\label{\detokenize{index:src.Views.Graph.computeVelocityStatistics}}\pysiglinewithargsret{\sphinxcode{\sphinxupquote{src.Views.Graph.}}\sphinxbfcode{\sphinxupquote{computeVelocityStatistics}}}{\emph{flightDict: dict}}{}
Computes statistics on the velocity points.
\begin{quote}\begin{description}
\item[{Parameters}] \leavevmode
\sphinxstyleliteralstrong{\sphinxupquote{flightDict}} \textendash{} Dictionary of flight data

\item[{Returns}] \leavevmode
Updated dictionary.

\end{description}\end{quote}

\end{fulllineitems}

\index{dimensionless\_jerk() (in module src.Views.Graph)@\spxentry{dimensionless\_jerk()}\spxextra{in module src.Views.Graph}}

\begin{fulllineitems}
\phantomsection\label{\detokenize{index:src.Views.Graph.dimensionless_jerk}}\pysiglinewithargsret{\sphinxcode{\sphinxupquote{src.Views.Graph.}}\sphinxbfcode{\sphinxupquote{dimensionless\_jerk}}}{\emph{movement: list}, \emph{fs: int}}{{ $\rightarrow$ float}}
Calculates the dimensionless jerk for a 1 dimensional array of points.
\begin{quote}\begin{description}
\item[{Parameters}] \leavevmode\begin{itemize}
\item {} 
\sphinxstyleliteralstrong{\sphinxupquote{movement}} \textendash{} The numpy array of points to calculate the jerk for. The array containing the movement speed profile. Doesn’t need to be numpy array but it MUST at least be a 1 dimensional list.

\item {} 
\sphinxstyleliteralstrong{\sphinxupquote{fs}} \textendash{} The sampling frequency of the data points.

\end{itemize}

\item[{Returns}] \leavevmode
The dimensionless jerk estimate of the given movement’s smoothness.

\end{description}\end{quote}

\end{fulllineitems}

\index{generateGraph() (in module src.Views.Graph)@\spxentry{generateGraph()}\spxextra{in module src.Views.Graph}}

\begin{fulllineitems}
\phantomsection\label{\detokenize{index:src.Views.Graph.generateGraph}}\pysiglinewithargsret{\sphinxcode{\sphinxupquote{src.Views.Graph.}}\sphinxbfcode{\sphinxupquote{generateGraph}}}{\emph{flightData: dict}, \emph{displayVelocity: bool}, \emph{t1: float}, \emph{t2: float}}{}
Driver function for generating the 3d graph of drone coordinates.
\begin{quote}\begin{description}
\item[{Parameters}] \leavevmode\begin{itemize}
\item {} 
\sphinxstyleliteralstrong{\sphinxupquote{flightData}} \textendash{} Dictionary containing flight data.

\item {} 
\sphinxstyleliteralstrong{\sphinxupquote{displayVelocity}} \textendash{} Boolean saying if velocity changes should be displayed on the graph.

\item {} 
\sphinxstyleliteralstrong{\sphinxupquote{t1}} \textendash{} Minimum time bound for plotting coordinates.

\item {} 
\sphinxstyleliteralstrong{\sphinxupquote{t2}} \textendash{} Maximum time bound for plotting coordinates.

\end{itemize}

\item[{Returns}] \leavevmode
The figure to display as the 3d graph.

\end{description}\end{quote}

\end{fulllineitems}

\index{log\_dimensionless\_jerk() (in module src.Views.Graph)@\spxentry{log\_dimensionless\_jerk()}\spxextra{in module src.Views.Graph}}

\begin{fulllineitems}
\phantomsection\label{\detokenize{index:src.Views.Graph.log_dimensionless_jerk}}\pysiglinewithargsret{\sphinxcode{\sphinxupquote{src.Views.Graph.}}\sphinxbfcode{\sphinxupquote{log\_dimensionless\_jerk}}}{\emph{movement: list}, \emph{fs: int}}{{ $\rightarrow$ float}}
Calculates the smoothness metric for the given speed profile using the log dimensionless jerk
metric.
\begin{quote}\begin{description}
\item[{Parameters}] \leavevmode\begin{itemize}
\item {} 
\sphinxstyleliteralstrong{\sphinxupquote{movement}} \textendash{} The numpy array of points to calculate the jerk for. The array containing the movement speed profile. Doesn’t need to be numpy array but it MUST at least be a 1 dimensional list.

\item {} 
\sphinxstyleliteralstrong{\sphinxupquote{fs}} \textendash{} The sampling frequency of the data points.

\end{itemize}

\item[{Returns}] \leavevmode
The dimensionless jerk estimate of the given movement’s smoothness.

\end{description}\end{quote}

\end{fulllineitems}

\index{velocityColors() (in module src.Views.Graph)@\spxentry{velocityColors()}\spxextra{in module src.Views.Graph}}

\begin{fulllineitems}
\phantomsection\label{\detokenize{index:src.Views.Graph.velocityColors}}\pysiglinewithargsret{\sphinxcode{\sphinxupquote{src.Views.Graph.}}\sphinxbfcode{\sphinxupquote{velocityColors}}}{\emph{flightDict: dict}}{}
Determines the color of the line segment between two points based on velocity values. The line color is determined
by the change in velocity of the drone between two points. The color of the line should be green if the
velocity point is greater than the previous velocity point, yellow if within 0.5 m/s, and red if less.
\begin{quote}\begin{description}
\item[{Parameters}] \leavevmode
\sphinxstyleliteralstrong{\sphinxupquote{flightDict}} \textendash{} Dictionary of flight data.

\item[{Returns}] \leavevmode
An array of color values to use when plotting line segments between points.

\end{description}\end{quote}

\end{fulllineitems}

\index{velocityPoints() (in module src.Views.Graph)@\spxentry{velocityPoints()}\spxextra{in module src.Views.Graph}}

\begin{fulllineitems}
\phantomsection\label{\detokenize{index:src.Views.Graph.velocityPoints}}\pysiglinewithargsret{\sphinxcode{\sphinxupquote{src.Views.Graph.}}\sphinxbfcode{\sphinxupquote{velocityPoints}}}{\emph{flightData: dict}}{}
Calculates the velocity of the drone between consecutive points for the entire flight.
\begin{quote}\begin{description}
\item[{Parameters}] \leavevmode
\sphinxstyleliteralstrong{\sphinxupquote{flightData}} \textendash{} Dictionary of flight Data.

\item[{Returns}] \leavevmode
Modified flightData dictionary.

\end{description}\end{quote}

\end{fulllineitems}



\chapter{Graph Test}
\label{\detokenize{index:module-src.Tests.Graph_Test}}\label{\detokenize{index:graph-test}}\index{src.Tests.Graph\_Test (module)@\spxentry{src.Tests.Graph\_Test}\spxextra{module}}\index{Graph\_Test (class in src.Tests.Graph\_Test)@\spxentry{Graph\_Test}\spxextra{class in src.Tests.Graph\_Test}}

\begin{fulllineitems}
\phantomsection\label{\detokenize{index:src.Tests.Graph_Test.Graph_Test}}\pysiglinewithargsret{\sphinxbfcode{\sphinxupquote{class }}\sphinxcode{\sphinxupquote{src.Tests.Graph\_Test.}}\sphinxbfcode{\sphinxupquote{Graph\_Test}}}{\emph{methodName='runTest'}}{}
This class is for testing the graphing functions that we are using to display the flight path
onto the UI. We want to make sure that what is being displayed is correct.
\index{test\_checkLegalInput() (src.Tests.Graph\_Test.Graph\_Test method)@\spxentry{test\_checkLegalInput()}\spxextra{src.Tests.Graph\_Test.Graph\_Test method}}

\begin{fulllineitems}
\phantomsection\label{\detokenize{index:src.Tests.Graph_Test.Graph_Test.test_checkLegalInput}}\pysiglinewithargsret{\sphinxbfcode{\sphinxupquote{test\_checkLegalInput}}}{}{{ $\rightarrow$ None}}
Test that legal and illegal coordinate points can be detected.
\begin{quote}\begin{description}
\item[{Returns}] \leavevmode
None

\end{description}\end{quote}

\end{fulllineitems}

\index{test\_computeVelocity() (src.Tests.Graph\_Test.Graph\_Test method)@\spxentry{test\_computeVelocity()}\spxextra{src.Tests.Graph\_Test.Graph\_Test method}}

\begin{fulllineitems}
\phantomsection\label{\detokenize{index:src.Tests.Graph_Test.Graph_Test.test_computeVelocity}}\pysiglinewithargsret{\sphinxbfcode{\sphinxupquote{test\_computeVelocity}}}{}{{ $\rightarrow$ None}}
Test that the velocity is computed as expected between two points (1,1,1) and (3,3,1) with timeDiff = 1.
\begin{quote}\begin{description}
\item[{Returns}] \leavevmode
None

\end{description}\end{quote}

\end{fulllineitems}

\index{test\_graphShows\_noError() (src.Tests.Graph\_Test.Graph\_Test method)@\spxentry{test\_graphShows\_noError()}\spxextra{src.Tests.Graph\_Test.Graph\_Test method}}

\begin{fulllineitems}
\phantomsection\label{\detokenize{index:src.Tests.Graph_Test.Graph_Test.test_graphShows_noError}}\pysiglinewithargsret{\sphinxbfcode{\sphinxupquote{test\_graphShows\_noError}}}{}{{ $\rightarrow$ None}}
Test that graph generates correctly with and with velocity changes shown
for data set of 100, then 200, then 800, then 1200 data points.
For each size, two tests are run. One test contains all legal inputs. Another test contains 80\% legal inputs.
Illegal coordinate points in file should not be included in “legalPoints” list in dictionary.
\begin{quote}\begin{description}
\item[{Returns}] \leavevmode
None

\end{description}\end{quote}

\end{fulllineitems}

\index{test\_readCoordinates\_size100\_allLegal() (src.Tests.Graph\_Test.Graph\_Test method)@\spxentry{test\_readCoordinates\_size100\_allLegal()}\spxextra{src.Tests.Graph\_Test.Graph\_Test method}}

\begin{fulllineitems}
\phantomsection\label{\detokenize{index:src.Tests.Graph_Test.Graph_Test.test_readCoordinates_size100_allLegal}}\pysiglinewithargsret{\sphinxbfcode{\sphinxupquote{test\_readCoordinates\_size100\_allLegal}}}{}{{ $\rightarrow$ None}}
Test that file containing 100 (x,y,z) points is read in correctly.
This test contains all legal inputs.
Illegal coordinate points in file should not be included in “legalPoints” list in dictionary.
\begin{quote}\begin{description}
\item[{Returns}] \leavevmode
None

\end{description}\end{quote}

\end{fulllineitems}

\index{test\_readCoordinates\_size100\_someLegal() (src.Tests.Graph\_Test.Graph\_Test method)@\spxentry{test\_readCoordinates\_size100\_someLegal()}\spxextra{src.Tests.Graph\_Test.Graph\_Test method}}

\begin{fulllineitems}
\phantomsection\label{\detokenize{index:src.Tests.Graph_Test.Graph_Test.test_readCoordinates_size100_someLegal}}\pysiglinewithargsret{\sphinxbfcode{\sphinxupquote{test\_readCoordinates\_size100\_someLegal}}}{}{{ $\rightarrow$ None}}
Test that file containing 100 (x,y,z) points is read in correctly.
This test contains 80\% legal inputs.
Illegal coordinate points in file should not be included in “legalPoints” list in dictionary.
\begin{quote}\begin{description}
\item[{Returns}] \leavevmode
None

\end{description}\end{quote}

\end{fulllineitems}

\index{test\_readCoordinates\_size1200\_allLegal() (src.Tests.Graph\_Test.Graph\_Test method)@\spxentry{test\_readCoordinates\_size1200\_allLegal()}\spxextra{src.Tests.Graph\_Test.Graph\_Test method}}

\begin{fulllineitems}
\phantomsection\label{\detokenize{index:src.Tests.Graph_Test.Graph_Test.test_readCoordinates_size1200_allLegal}}\pysiglinewithargsret{\sphinxbfcode{\sphinxupquote{test\_readCoordinates\_size1200\_allLegal}}}{}{{ $\rightarrow$ None}}
Test that file containing 1200 (x,y,z) points is read in correctly. This contains all legal inputs.
Illegal coordinate points in file should not be included in “legalPoints” list in dictionary.
\begin{quote}\begin{description}
\item[{Returns}] \leavevmode
None

\end{description}\end{quote}

\end{fulllineitems}

\index{test\_readCoordinates\_size1200\_someLegal() (src.Tests.Graph\_Test.Graph\_Test method)@\spxentry{test\_readCoordinates\_size1200\_someLegal()}\spxextra{src.Tests.Graph\_Test.Graph\_Test method}}

\begin{fulllineitems}
\phantomsection\label{\detokenize{index:src.Tests.Graph_Test.Graph_Test.test_readCoordinates_size1200_someLegal}}\pysiglinewithargsret{\sphinxbfcode{\sphinxupquote{test\_readCoordinates\_size1200\_someLegal}}}{}{{ $\rightarrow$ None}}
Test that file containing 1200 (x,y,z) points is read in correctly. This contains 80\% legal inputs.
Illegal coordinate points in file should not be included in “legalPoints” list in dictionary.
\begin{quote}\begin{description}
\item[{Returns}] \leavevmode
None

\end{description}\end{quote}

\end{fulllineitems}

\index{test\_readCoordinates\_size200\_allLegal() (src.Tests.Graph\_Test.Graph\_Test method)@\spxentry{test\_readCoordinates\_size200\_allLegal()}\spxextra{src.Tests.Graph\_Test.Graph\_Test method}}

\begin{fulllineitems}
\phantomsection\label{\detokenize{index:src.Tests.Graph_Test.Graph_Test.test_readCoordinates_size200_allLegal}}\pysiglinewithargsret{\sphinxbfcode{\sphinxupquote{test\_readCoordinates\_size200\_allLegal}}}{}{{ $\rightarrow$ None}}
Test that file containing 200 (x,y,z) points is read in correctly. This contains all legal inputs.
Illegal coordinate points in file should not be included in “legalPoints” list in dictionary.
\begin{quote}\begin{description}
\item[{Returns}] \leavevmode
None

\end{description}\end{quote}

\end{fulllineitems}

\index{test\_readCoordinates\_size200\_someLegal() (src.Tests.Graph\_Test.Graph\_Test method)@\spxentry{test\_readCoordinates\_size200\_someLegal()}\spxextra{src.Tests.Graph\_Test.Graph\_Test method}}

\begin{fulllineitems}
\phantomsection\label{\detokenize{index:src.Tests.Graph_Test.Graph_Test.test_readCoordinates_size200_someLegal}}\pysiglinewithargsret{\sphinxbfcode{\sphinxupquote{test\_readCoordinates\_size200\_someLegal}}}{}{{ $\rightarrow$ None}}
Test that file containing 200 (x,y,z) points is read in correctly. This contains 80\% legal inputs.
Illegal coordinate points in file should not be included in “legalPoints” list in dictionary.
\begin{quote}\begin{description}
\item[{Returns}] \leavevmode
None

\end{description}\end{quote}

\end{fulllineitems}

\index{test\_readCoordinates\_size600\_allLegal() (src.Tests.Graph\_Test.Graph\_Test method)@\spxentry{test\_readCoordinates\_size600\_allLegal()}\spxextra{src.Tests.Graph\_Test.Graph\_Test method}}

\begin{fulllineitems}
\phantomsection\label{\detokenize{index:src.Tests.Graph_Test.Graph_Test.test_readCoordinates_size600_allLegal}}\pysiglinewithargsret{\sphinxbfcode{\sphinxupquote{test\_readCoordinates\_size600\_allLegal}}}{}{{ $\rightarrow$ None}}
Test that file containing 600 (x,y,z) points is read in correctly. This contains all legal inputs.
Illegal coordinate points in file should not be included in “legalPoints” list in dictionary.
\begin{quote}\begin{description}
\item[{Returns}] \leavevmode
None

\end{description}\end{quote}

\end{fulllineitems}

\index{test\_readCoordinates\_size600\_someLegal() (src.Tests.Graph\_Test.Graph\_Test method)@\spxentry{test\_readCoordinates\_size600\_someLegal()}\spxextra{src.Tests.Graph\_Test.Graph\_Test method}}

\begin{fulllineitems}
\phantomsection\label{\detokenize{index:src.Tests.Graph_Test.Graph_Test.test_readCoordinates_size600_someLegal}}\pysiglinewithargsret{\sphinxbfcode{\sphinxupquote{test\_readCoordinates\_size600\_someLegal}}}{}{{ $\rightarrow$ None}}
Test that file containing 600 (x,y,z) points is read in correctly. This contains 80\% legal inputs.
Illegal coordinate points in file should not be included in “legalPoints” list in dictionary.
\begin{quote}\begin{description}
\item[{Returns}] \leavevmode
None

\end{description}\end{quote}

\end{fulllineitems}

\index{test\_readCoordinates\_small\_illegal() (src.Tests.Graph\_Test.Graph\_Test method)@\spxentry{test\_readCoordinates\_small\_illegal()}\spxextra{src.Tests.Graph\_Test.Graph\_Test method}}

\begin{fulllineitems}
\phantomsection\label{\detokenize{index:src.Tests.Graph_Test.Graph_Test.test_readCoordinates_small_illegal}}\pysiglinewithargsret{\sphinxbfcode{\sphinxupquote{test\_readCoordinates\_small\_illegal}}}{}{{ $\rightarrow$ None}}
Test that coordinates from a small data file are read in correctly.
\begin{quote}\begin{description}
\item[{Returns}] \leavevmode
None

\end{description}\end{quote}

\end{fulllineitems}

\index{test\_readCoordinates\_small\_legal() (src.Tests.Graph\_Test.Graph\_Test method)@\spxentry{test\_readCoordinates\_small\_legal()}\spxextra{src.Tests.Graph\_Test.Graph\_Test method}}

\begin{fulllineitems}
\phantomsection\label{\detokenize{index:src.Tests.Graph_Test.Graph_Test.test_readCoordinates_small_legal}}\pysiglinewithargsret{\sphinxbfcode{\sphinxupquote{test\_readCoordinates\_small\_legal}}}{}{{ $\rightarrow$ None}}
Test that coordinates from a small data file are read in correctly.
\begin{quote}\begin{description}
\item[{Returns}] \leavevmode
None

\end{description}\end{quote}

\end{fulllineitems}

\index{test\_smoothnessComputes() (src.Tests.Graph\_Test.Graph\_Test method)@\spxentry{test\_smoothnessComputes()}\spxextra{src.Tests.Graph\_Test.Graph\_Test method}}

\begin{fulllineitems}
\phantomsection\label{\detokenize{index:src.Tests.Graph_Test.Graph_Test.test_smoothnessComputes}}\pysiglinewithargsret{\sphinxbfcode{\sphinxupquote{test\_smoothnessComputes}}}{}{{ $\rightarrow$ None}}
Test that smoothness function returns a number when inputted data set of 100, then 200, then 800,
then 1200 data points. For each size, two tests are run.
One test contains all legal inputs. Another test contains 80\% legal inputs.
Illegal coordinate points in file should not be included in “legalPoints” list in dictionary.
\begin{quote}\begin{description}
\item[{Returns}] \leavevmode
None

\end{description}\end{quote}

\end{fulllineitems}

\index{test\_smoothnessValues() (src.Tests.Graph\_Test.Graph\_Test method)@\spxentry{test\_smoothnessValues()}\spxextra{src.Tests.Graph\_Test.Graph\_Test method}}

\begin{fulllineitems}
\phantomsection\label{\detokenize{index:src.Tests.Graph_Test.Graph_Test.test_smoothnessValues}}\pysiglinewithargsret{\sphinxbfcode{\sphinxupquote{test\_smoothnessValues}}}{}{{ $\rightarrow$ None}}
Test that smoothness function returns expected number.
\begin{quote}\begin{description}
\item[{Returns}] \leavevmode
None

\end{description}\end{quote}

\end{fulllineitems}

\index{test\_velocityColors() (src.Tests.Graph\_Test.Graph\_Test method)@\spxentry{test\_velocityColors()}\spxextra{src.Tests.Graph\_Test.Graph\_Test method}}

\begin{fulllineitems}
\phantomsection\label{\detokenize{index:src.Tests.Graph_Test.Graph_Test.test_velocityColors}}\pysiglinewithargsret{\sphinxbfcode{\sphinxupquote{test\_velocityColors}}}{}{{ $\rightarrow$ None}}
Test that the correct color is assigned to the graph segment between consecutive velocity values.
\begin{quote}\begin{description}
\item[{Returns}] \leavevmode
None

\end{description}\end{quote}

\end{fulllineitems}

\index{test\_velocityColorsComputes\_correctSize() (src.Tests.Graph\_Test.Graph\_Test method)@\spxentry{test\_velocityColorsComputes\_correctSize()}\spxextra{src.Tests.Graph\_Test.Graph\_Test method}}

\begin{fulllineitems}
\phantomsection\label{\detokenize{index:src.Tests.Graph_Test.Graph_Test.test_velocityColorsComputes_correctSize}}\pysiglinewithargsret{\sphinxbfcode{\sphinxupquote{test\_velocityColorsComputes\_correctSize}}}{}{{ $\rightarrow$ None}}
Test that velocityColors returns array of correct size when inputted data set of 100, then 200, then 800,
then 1200 data points. For each size, two tests are run.
One test contains all legal inputs. Another test contains 80\% legal inputs.
Illegal coordinate points in file should not be included in “legalPoints” list in dictionary.
\begin{quote}\begin{description}
\item[{Returns}] \leavevmode
None

\end{description}\end{quote}

\end{fulllineitems}

\index{test\_velocityComputes\_correctSize() (src.Tests.Graph\_Test.Graph\_Test method)@\spxentry{test\_velocityComputes\_correctSize()}\spxextra{src.Tests.Graph\_Test.Graph\_Test method}}

\begin{fulllineitems}
\phantomsection\label{\detokenize{index:src.Tests.Graph_Test.Graph_Test.test_velocityComputes_correctSize}}\pysiglinewithargsret{\sphinxbfcode{\sphinxupquote{test\_velocityComputes\_correctSize}}}{}{{ $\rightarrow$ None}}
Test that velocityPoints returns array of correct size when inputted data set of 100, 200, then 800,
then 1200 data points. For each size, two tests are run.
One test contains all legal inputs. Another test contains 80\% legal inputs.
Illegal coordinate points in file should not be included in “legalPoints” list in dictionary.
\begin{quote}\begin{description}
\item[{Returns}] \leavevmode
None

\end{description}\end{quote}

\end{fulllineitems}

\index{test\_velocityPoints() (src.Tests.Graph\_Test.Graph\_Test method)@\spxentry{test\_velocityPoints()}\spxextra{src.Tests.Graph\_Test.Graph\_Test method}}

\begin{fulllineitems}
\phantomsection\label{\detokenize{index:src.Tests.Graph_Test.Graph_Test.test_velocityPoints}}\pysiglinewithargsret{\sphinxbfcode{\sphinxupquote{test\_velocityPoints}}}{}{{ $\rightarrow$ None}}
Test that velocity between consecutive points is computed as expected.
\begin{quote}\begin{description}
\item[{Returns}] \leavevmode
None

\end{description}\end{quote}

\end{fulllineitems}


\end{fulllineitems}



\chapter{Import File Test}
\label{\detokenize{index:module-src.Tests.ImportFile_Test}}\label{\detokenize{index:import-file-test}}\index{src.Tests.ImportFile\_Test (module)@\spxentry{src.Tests.ImportFile\_Test}\spxextra{module}}\index{ImportFileTests (class in src.Tests.ImportFile\_Test)@\spxentry{ImportFileTests}\spxextra{class in src.Tests.ImportFile\_Test}}

\begin{fulllineitems}
\phantomsection\label{\detokenize{index:src.Tests.ImportFile_Test.ImportFileTests}}\pysiglinewithargsret{\sphinxbfcode{\sphinxupquote{class }}\sphinxcode{\sphinxupquote{src.Tests.ImportFile\_Test.}}\sphinxbfcode{\sphinxupquote{ImportFileTests}}}{\emph{methodName='runTest'}}{}
This test class is used the test the import and export functions to ensure we can successfully
save and reload past flights.
\index{test\_import() (src.Tests.ImportFile\_Test.ImportFileTests method)@\spxentry{test\_import()}\spxextra{src.Tests.ImportFile\_Test.ImportFileTests method}}

\begin{fulllineitems}
\phantomsection\label{\detokenize{index:src.Tests.ImportFile_Test.ImportFileTests.test_import}}\pysiglinewithargsret{\sphinxbfcode{\sphinxupquote{test\_import}}}{}{{ $\rightarrow$ None}}
Test that .flight file generated from exporting flight data can be imported successfully.
All data members should exist, and no extra keys should be in the file.
\begin{quote}\begin{description}
\item[{Returns}] \leavevmode
None

\end{description}\end{quote}

\end{fulllineitems}


\end{fulllineitems}



\chapter{Indices and tables}
\label{\detokenize{index:indices-and-tables}}\begin{itemize}
\item {} 
\DUrole{xref,std,std-ref}{genindex}

\item {} 
\DUrole{xref,std,std-ref}{modindex}

\item {} 
\DUrole{xref,std,std-ref}{search}

\end{itemize}


\renewcommand{\indexname}{Python Module Index}
\begin{sphinxtheindex}
\let\bigletter\sphinxstyleindexlettergroup
\bigletter{s}
\item\relax\sphinxstyleindexentry{src.Controllers.Exceptions}\sphinxstyleindexpageref{index:\detokenize{module-src.Controllers.Exceptions}}
\item\relax\sphinxstyleindexentry{src.Controllers.OpenCVThreadedController}\sphinxstyleindexpageref{index:\detokenize{module-src.Controllers.OpenCVThreadedController}}
\item\relax\sphinxstyleindexentry{src.Controllers.PhoneController}\sphinxstyleindexpageref{index:\detokenize{module-src.Controllers.PhoneController}}
\item\relax\sphinxstyleindexentry{src.Controllers.Program\_Controller}\sphinxstyleindexpageref{index:\detokenize{module-src.Controllers.Program_Controller}}
\item\relax\sphinxstyleindexentry{src.Export.ExportFile}\sphinxstyleindexpageref{index:\detokenize{module-src.Export.ExportFile}}
\item\relax\sphinxstyleindexentry{src.Export.ImportFile}\sphinxstyleindexpageref{index:\detokenize{module-src.Export.ImportFile}}
\item\relax\sphinxstyleindexentry{src.Tests.Graph\_Test}\sphinxstyleindexpageref{index:\detokenize{module-src.Tests.Graph_Test}}
\item\relax\sphinxstyleindexentry{src.Tests.ImportFile\_Test}\sphinxstyleindexpageref{index:\detokenize{module-src.Tests.ImportFile_Test}}
\item\relax\sphinxstyleindexentry{src.Views.Graph}\sphinxstyleindexpageref{index:\detokenize{module-src.Views.Graph}}
\item\relax\sphinxstyleindexentry{src.Views.View\_LoadingScreen}\sphinxstyleindexpageref{index:\detokenize{module-src.Views.View_LoadingScreen}}
\item\relax\sphinxstyleindexentry{src.Views.View\_ReportScreen}\sphinxstyleindexpageref{index:\detokenize{module-src.Views.View_ReportScreen}}
\item\relax\sphinxstyleindexentry{src.Views.View\_StartupScreen}\sphinxstyleindexpageref{index:\detokenize{module-src.Views.View_StartupScreen}}
\item\relax\sphinxstyleindexentry{src.Views.View\_TrackingScreen}\sphinxstyleindexpageref{index:\detokenize{module-src.Views.View_TrackingScreen}}
\item\relax\sphinxstyleindexentry{src.Views.View\_VerifySetupScreen}\sphinxstyleindexpageref{index:\detokenize{module-src.Views.View_VerifySetupScreen}}
\end{sphinxtheindex}

\renewcommand{\indexname}{Index}
\printindex
\end{document}